\documentclass[12pt]{article}
\usepackage{preamble}

\pagestyle{fancy}
\fancyhead[LO,LE]{Специальные разделы \\ высшей математики}
\fancyhead[CO,CE]{17.05.2024}
\fancyhead[RO,RE]{Лекции Далевской О. П.}


\begin{document}
    \Mem ЛДУ$\displaystyle _2$

    1) Решим $\displaystyle y^{\prime\prime} + py^\prime + qy = 0$ (ХрУ \Cat: $\displaystyle \lambda^2 + p \lambda + q = 0$)

    ФСР для всех случаев:

    1* $\displaystyle \lambda_1 \neq \lambda_2 \in \Real \to \Set{e^{\lambda_1 x}, e^{\lambda_2 x}}$

    2* $\displaystyle \lambda_1 = \lambda_2 = \lambda \in \Real \to \Set{e^{\lambda x}, xe^{\lambda x}}$

    3* $\displaystyle \lambda_{1,2} = \alpha \pm i \beta \to \Set{e^{\alpha x} \cos \beta x, e^{\alpha x} \sin \beta x}$

    $\displaystyle \overline{y} = l_{\Set{\text{ФСР}}}$

    2) Изначально $\displaystyle y^{\prime\prime} + py^\prime + qy = f(x)$

    Доказали: $\displaystyle y(x) = \overline{y} + y^*$, где $\displaystyle \overline{y} = \sum_{i=1}^n C_i y_i$ - вектора из ФСР, а $\displaystyle y^*$ - частное решение (какое-либо) ЛНДУ

    \Nota Рассмотрим два метода поиска $\displaystyle y^*$ для ЛДУ$\displaystyle _2$

    1* Метод неопределенных коэффициентов для случая специальной правой части

    2* Метод (Лагранжа) вариации произвольных постоянных (универсальный)

    \vspace{10mm}

    1* \textbf{СПЧ}

    \Ex $\displaystyle y^{\prime\prime} - 3y^\prime + 3y = 2e^{3x} \quad (\heartsuit)$

    Наводящие соображения: Заметим, что $\displaystyle y = e^{ax}$ не меняет свой вид при дифференцировании,
    так же как и $\displaystyle y = P_n(x)$, $y = A\cos bx + B\cos bx$

    Имеет смысл искать частные решения $(\heartsuit)$ в виде $\displaystyle y = Ae^{3x}$

    $\displaystyle (Ae^3x)^{\prime\prime} - 3(Ae^{3x})^\prime + 2Ae^{3x} = 2e^{3x}$

    $9A - 9A + 2A = 2 \Longrightarrow A = 1$, то есть $\displaystyle y^* = e^{3x}$

    \Nota Если правая часть содержит произведения $\displaystyle e^{ax}, P_n(x), \cos bx, \sin bx$, то $\displaystyle y^*$ ищем в виде ПЧ

    \Def СПЧ: $\displaystyle f(x) = e^{ax} (P_n(x)\cos bx + Q_m (x)\sin bx)$ (обозначим $k = a \pm ib$)

    \underline{Частные случаи}:

    1) $\displaystyle f(x) = P_n(x) e^{ax} \quad (b = 0)$

    2) $f(x) = A\cos b x + B \sin bx$ - гармоника $\quad (a = 0, n = m = 0)$

    3) $\displaystyle f(x) = P_n(x) \quad (a = b = 0)$

    Метод: Решение ищется в виде $\displaystyle y^* = e^{ax} (\overline{P}_l \cos bx + \overline{Q}_l (x) \sin bx)$,
    где $a, b$ - коэфф. СПЧ, $\displaystyle l = \max(m, n), \overline{P}_l, \overline{Q}_l$ - многочлены в неопр. коэфф

    \ExN{1} $\displaystyle \heartsuit \quad y^{\prime\prime} - 3y^\prime + 3y = 2e^{3x} = e^{3x} (2 \cos 0x) \quad (k = 3 \pm 0 = 3)$

    $\displaystyle y^* = e^{3x} (\overline{P}_{l = 0} (x) \cos 0x) = e^{3x} \cdot A$

    \ExN{2} Однако!

    $\displaystyle y^{\prime\prime} - 3y^\prime + 2y = e^{2x}$ (!)

    СПЧ: $\displaystyle e^{2x} = e^{2x} (1 \cos 0x + B \sin 0x) \quad k = a \pm ib$

    \begin{rcases*}
    $\displaystyle y^* = Ae^{2x}$ \\

    $\displaystyle y^{*\prime} = 2Ae^{2x}$ \\

    $\displaystyle y^{*\prime\prime} = 4Ae^{2x}$ \\
    \end{rcases*} ДУ $\displaystyle \longrightarrow \begin{matrix}4Ae^{2x} - 6Ae^{2x} + 2Ae^{2x} = e^{2x} \\ 4A - 6A + 2A = 1 \\ 0A = 1\end{matrix}$ - {\Huge 🤯}

    Нельзя найти $A$

    Решим ХрУ 😼: $\displaystyle \lambda_2 - 3\lambda + 2 = 0 \Longrightarrow \lambda_1 = 1, \lambda_2 = 2$

    Внимание! Число $k$, соответствующее СПЧ, равно ХрУ \Cat

    \vspace{5mm}

    Исследуем ситуацию на примере СПЧ $\displaystyle f(x) = P_n(x) e^{ax}$

    \underline{Проблема} $\displaystyle y^{\prime\prime} + py^\prime + qy = P_n(x)e^{ax}$

    ХрУ \Cat: $\displaystyle \lambda^2 + p\lambda + q = 0 \Longrightarrow \lambda_{1,2}$ - корни

    \vspace{5mm}
    Ищем $\displaystyle y^* = \overline{P}_n(x) e^{ax}$

    $\displaystyle y^{*\prime} = \overline{P}_{n - 1} (x) e^{ax} + a\overline{P}_n(x) e^{ax}$

    $\displaystyle y^{*\prime} = \overline{P}_{n - 2} (x) e^{ax} + 2a\overline{P}_{n - 1} (x) e^{ax} + a^2\overline{P}_n(x) e^{ax}$

    \vspace{5mm}
    Получаем:

    $\displaystyle \overline{P}_{n - 2} (x) e^{ax} + 2a\overline{P}_{n - 1} (x) e^{ax} + a^2\overline{P}_n(x) e^{ax} + (\overline{P}_{n - 1} (x) e^{ax} + a\overline{P}_n(x) e^{ax})p + \overline{P}_n(x) e^{ax} q $

    $\displaystyle \overline{P}_{n - 2} (x) e^{ax} + (2a + p)\overline{P}_{n - 1} (x) e^{ax} + (a^2 + pa + q)\overline{P}_n(x) e^{ax} = P_n(x) e^{ax}$

    $\displaystyle \overline{P}_{n - 2} (x) + (2a + p)\overline{P}_{n - 1} (x) + (a^2 + pa + q)\overline{P}_n(x) = P_n(x)$

    Заметим, что если $a$ - корень ХрУ \Cat, то есть $\displaystyle a \pm ib = a = k = \lambda_i$ (пусть 1-ой кратности), то $\displaystyle a^2 + pa + q = 0$ и степень
    левой части понижается до $n - 1$

    Если $a$ - корень ХрУ \Cat 2-ой кратности, то есть $\displaystyle a^2 + pa + q = \left(a + \frac{p}{2}\right)^2 = 0 \Longleftrightarrow 2a + p = 0$, то степень левой части понижается на $2$

    Чтобы сделать уравнение для $\displaystyle \overline{P}_n$ решаемым, домножим $\displaystyle y^*$ на $\displaystyle x^r$, где $r$ - число совпадений $k = a \pm ib$ с корнем ХрУ $\displaystyle \lambda_i$ (или кратность $\displaystyle \lambda_i$, с которым совпадает $k$)

    \vspace{5mm}
    Метод (окончательно): $\displaystyle y^{\prime\prime} + py^\prime + qy = e^{ax} (P_n(x)\cos bx + Q_m (x)\sin bx), \quad \lambda_{1,2}$ - корни ХрУ \Cat, $k = a \pm ib$

    $\displaystyle y^* = x^r e^{ax} (\overline{P}_l (x)\cos bx + \overline{Q}_l (x) \sin bx), \quad l = \max(m, n)$

    \vspace{5mm}
    Обобщение для ЛДУ$\displaystyle _n$

    $\displaystyle y^{(n)} + p_1 y^{(n - 1)} + \dots + p_n y = f(x)$

    ХрУ \Cat: $\displaystyle \lambda^n + p_1 \lambda^{n - 1} + \dots + p_n = 0$

    \vspace{5mm}
    \underline{Правило} построения ФСР для $\overline{y}$ - общее решение однородного ДУ

    1) Всякому $\displaystyle \lambda_i$ - одиночному $\Real$-корню ХрУ сопоставляем $\displaystyle y_i = e^{\lambda_i x}$

    2) \Real-корню $\lambda$ кратности $s$ сопоставляем набор $\displaystyle \Set{y_1, y_2, \dots, y_s} = \Set{e^{\lambda x}, xe^{\lambda x}, \dots, x^{s - 1} e^{\lambda x}}$

    3) Всякой одиночной паре $\displaystyle \lambda_{j_1,j_2} = \alpha_j \pm i\beta_j$ соотвветствует пара $\displaystyle \Set{e^{\alpha x} \cos\beta x, e^{\alpha x} \sin\beta x}$

    4) $\mathbb{C}$-паре $\lambda = \alpha \pm i\beta$ кратности $t$ соответствует набор $\displaystyle \{e^{\alpha x} \cos \beta x, e^{\alpha x} \sin \beta x, x e^{\alpha x} \cos \beta x, \dots, x^{t - 1}e^{\alpha x} \cos\beta x, x^{t - 1}e^{\alpha x} \sin\beta x\}$

    \Nota количество векторов $\displaystyle y_i$ в ФСР равно порядку $n$ ДУ

    \underline{СПЧ} $\displaystyle y^* = x^r e^{ax} (\dots)$, где $r$ - кратность $\Real$-корня или $\mathbb{C}$-пары, с которыми совпадает $k = a \pm ib$

    \Ex Вернемся к $\displaystyle y^{\prime\prime} - 3y^\prime + 2y = e^{2x}$

    $\displaystyle \begin{rcases*}
    y^* = Ax^1 e^{2x} \\

    y^{*\prime} = Ae^{2x} 2Axe^{2x} \\

    y^{*\prime\prime} = 2Ae^{2x} + 2Ae^{2x} + 4Axe^{2x}
    \end{rcases*} \to (4 - 6 + 2) Axe^{2x} + (4 - 3) Ae^{2x} = e^{2x} \quad A = 1$

    $\displaystyle y(x) = C_1 e^{2x} + C_2 e^{2x} + xe^{2x}$

    \vspace{10mm}

    2* \textbf{Лагранжа}

    \Mem ЛДУ$\displaystyle _1$: $\displaystyle y^\prime + py = f(x)$

    1) ЛОДУ - $\displaystyle y^\prime + py = 0 \to \overline{y} = Cy_0$ - ФСР

    2) ЛНДУ - $\displaystyle y(x) = C(x)y_0 \longrightarrow C^\prime(x) y_0 = f(x) \to C(x)$

    \Nota Введем аналогичный метод для ЛДУ$\displaystyle _2$

    1 этап) $\displaystyle ^{\prime\prime} + py^\prime + qy = 0$ - ЛОДУ, $\displaystyle \quad \lambda_{1, 2}$ - корни, соответствующие ФСР $\displaystyle \Set{y_1, y_2}$

    $\displaystyle \overline{y}(x) = C_1 y_1 + C_2 y_2$

    2 этап) Варьируем $\displaystyle C_1$ и $\displaystyle C_2$, но теперь нужны два условия для их определения. Одним является ДУ

    \Ex $\displaystyle y^{\prime\prime} - 3y^\prime + 2y = 2e^{3x}$

    $\displaystyle \overline{y} = C_1 e^x + C_2 e^{2x}$

    $\displaystyle y(x) = C_1(x)e^x + C_2(X)e^{2x} = C_1 e^x + C_2 e^{2x} + y^*$

    $\displaystyle (g(x) + C_1)e^x + (h(x) + C_2)e^{2x} = C_1 e^x + C_2 e^{2x} + g(x)e^x + h(x)e^{2x}$

    Подберем $g, h$: $\displaystyle \underset{g}{\undergroup{\frac{e^{2x}}{2}}}e^x + \underset{h}{\undergroup{\frac{e^x}{2}}}e^{2x} = e^{3x}$ или
    $\displaystyle \underset{g}{\undergroup{-e^{2x}}} e^x + \underset{g}{\undergroup{2e^x}} e^{2x} = e^{3x}$

    Заметим, что $\displaystyle C_1^\prime(x)$ во втором случае $\displaystyle g^\prime = -2e^{2x}$, а $\displaystyle C_2^\prime = 2e^x$

    Тогда $\displaystyle C_1^\prime(x) e^x + C^\prime_2 (x) e^{2x} = -2e^{3x} + 2e^{3x} = 0$

    \Nota Подставим $\displaystyle y(x) = C_1 (x) y_1 + C_2 (x) y_2$ в ДУ

    \underline{Метод} $\displaystyle y^\prime(x) = C^\prime_1(x) y_1 + C_1(x)y^\prime_1 + C^\prime_2(x) y_2 + C_2(x)y^\prime_2$

    Требуем $\displaystyle C^\prime_1 y_1 + C_2^\prime y_2 = 0$

    $\displaystyle y^{\prime\prime}(x) = C_1^\prime (x) y_1^\prime + C_1 (x) y_1^{\prime\prime} + C_2^\prime (x) y_2^\prime C_2 (x) y_2^{\prime\prime}$

    $\displaystyle C_1^\prime (x) y_1^\prime + C_1 (x) y_1^{\prime\prime} + C_2^\prime (x) y_2^\prime C_2 (x) y_2^{\prime\prime} + pC_1(x)y^\prime_1 + pC_2(x)y^\prime_2 + qC_1(x)y_1 + qC_2(x)y_2 = f(x)$

    $\displaystyle \underset{=\ 0}{C_1(x)Ly_1} + \underset{=\ 0}{C_2(x)Ly_2} + C_1^\prime(x)y_1^\prime + C_2^\prime(x)y_2^\prime = f(x)$

    \underline{Итак}, Система для определения $\displaystyle C_1(x), C_2(x)$:
    \begin{cases}
        C^\prime_1(x)y_1 + C^\prime_2(x)y_2 = 0 \\
        C_1^\prime(x)y_1^\prime + C_2^\prime(x)y_2^\prime = f(x)
    \end{cases}

    $\displaystyle \underset{ = W}{\undergroup{\begin{pmatrix}y_1 & y_2 \\ y^\prime_1 & y^\prime_2\end{pmatrix}}}\begin{pmatrix}C^\prime_1(x) \\ C^\prime_2(x)\end{pmatrix} = \begin{pmatrix}0 \\ f(x)\end{pmatrix} \stackrel{\text{Крамер}}{\longrightarrow} \begin{matrix}C_1^\prime(x) = \frac{W_1}{W} \\ C_2^\prime(x) = \frac{W_2}{W}\end{matrix}$

    \Nota Обобщив метод на $n$-ый порядок систему, получим

    $\displaystyle \begin{pmatrix}y_1 & \dots & y_n \\ \vdots & \ddots & \vdots \\ y^{(n - 2)}_1 & \dots & y^{(n - 2)}_n \\ y^{(n - 1)}_1 & \dots & y^{(n - 1)}_n\end{pmatrix}\begin{pmatrix}C^\prime_1(x) \\ \vdots \\ C^\prime_{n - 1}(x) \\ C^\prime_n(x)\end{pmatrix} = \begin{pmatrix}0 \\ \vdots \\ 0 \\ f(x)\end{pmatrix}$

    ? Доказать, что $\displaystyle \overline{y} + y^*$ - общее решение ЛНДУ

\end{document}

