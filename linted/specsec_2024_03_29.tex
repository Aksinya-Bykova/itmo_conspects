\documentclass[12pt]{article}
\usepackage{preamble}

\pagestyle{fancy}
\fancyhead[LO,LE]{Специальные разделы \\ высшей математики}
\fancyhead[CO,CE]{29.03.2024}
\fancyhead[RO,RE]{Лекции Далевской О. П.}


\begin{document}
    \Th $\displaystyle A^\prime = T_{e\to e^\prime} A T^{-1}_{e\to e^\prime}$

    \Nota $C = A + \lambda B$

    Следствия:

    1) $\displaystyle TCT^{-1} = T (A + \lambda B) T^{-1} = T A T^{-1} + \lambda T B T^{-1}$

    2) $\displaystyle B = I \quad T B T^{-1} = T I T^{-1} = I$, т. к. $\displaystyle TI = T, T T^{-1} = I$

    3) $\displaystyle \det A^{-1} = \det (T A T^{-1}) = \det T \det A \det T^{-1} = \det A \cdot 1$

    \Nota То есть характеристика нашего объекта - инвариант при преобразовании $T$

    \Def Матрица $A$ называется ортогональной если $\displaystyle A^{-1} = A^T$

    Следствие: $\displaystyle AA^{-1} = AA^T = I$

    $\displaystyle \begin{pmatrix}
         a_{11} & a_{12} & \dots & a_{1n} \\
         a_{21} & a_{22} & \dots & a_{2n} \\
         \vdots & \vdots & \ddots & \vdots \\
         a_{n1} & a_{n2} & \dots & a_{nn} \\
    \end{pmatrix} \cdots \begin{pmatrix}
         a_{11} & a_{21} & \dots & a_{n1} \\
         a_{12} & a_{22} & \dots & a_{n2} \\
         \vdots & \vdots & \ddots & \vdots \\
         a_{1n} & a_{2n} & \dots & a_{nn} \\
    \end{pmatrix} = \begin{pmatrix}
         1 & 0 & \dots & 0 \\
         0 & 1 & \dots & 0 \\
         \vdots & \vdots & \ddots & \vdots \\
         0 & 0 & \dots & 1 \\
    \end{pmatrix}$

    $\displaystyle \forall i \Sigma^n_{j=1} a_{ij} a_{ij} = (A_i, A_i) = 1$
    $\displaystyle \forall i, j (i \neq j) \Sigma^n_{kk=1} a_{ik} a_{jk} = (A_i, A_j) = 0$

    В общем $\displaystyle (A_i, A_j) = \begin{sqcases}1, i = j \\ 0, i \neq j \end{sqcases}$

    \Def Оператор $\mathcal{A}$ называется ортогональным, если его матрица ортогональна

    ? $A$ ортогональна в каком-либо базисе или во всех?

    Свойство. $\mathcal{A}$ - ортогонален, то $\det A = \pm 1$ (следует из определения $\displaystyle \det(AA^T) = \det^2(A) = \det(I)$)

    \Th $\displaystyle T_{e\to e^\prime}$ - преобразование координат в $\displaystyle V^n$. Тогда $T$ - ортогональный оператор

    Базис $e$ - ортонормированный базис

    $\Box \quad \sqsupset $ в базисе $e$ матрица $\displaystyle T = \begin{pmatrix}
          \tau_{11} & \dots & \tau_{1n} \\
          \vdots & \ddots & \vdots \\
          \tau_{n1} & \dots & \tau_{nn} \\
    \end{pmatrix}$ - неортогональна

    Тогда $\displaystyle e_1^\prime = \Sigma_{i=1}^n \tau_{1i} e_i \quad \Big| \cdot e_1^\prime$

    $\displaystyle 1 = (e_1^\prime, e_1^\prime) = (\Sigma_{i=1}^n \tau_{1i} e_i)^2 =
    \tau^2_{11} e^2_1 + \tau_{11} e_1 \tau_{12} e_2 + \dots = \tau_{11}^2 + \dots + \tau_{1n}^2 = 1$ - то есть строка - единичный вектор

    $\displaystyle 0 = (e_1^\prime, e_2^\prime) = (\tau_{11} e_1 + \tau_{12}e_1 + \dots) \cdot
    (\tau_{21}e_1 + \tau_{22}e_2 + \dots) = $ произведение 1-ой строки на 2-ую, то есть строки ортогональны

    Таким образом, матрица $T$ - ортогональна

    \Nota Тогда $\displaystyle A^\prime = T A T^{-1} = T A T^T$

    \section[p2\_7]{2.7. Собственные векторы и значения оператора}

    \Def Инвариантное подпространство оператора $\mathcal{A} : V \rightarrow V$ -
    это $\displaystyle U = \Set{x \in V_1 \in V | \mathcal{A}x \in V_1}$

    \Ex $\displaystyle V = \mathcal{P}_n(t)$ - пространство многочленов степени $\leq n$ на $[a; b]$, $\displaystyle \mathcal{D} = \frac{d}{dt}$

    \Nota $Ker \mathcal{A}, Im \mathcal{A}$ - инвариантные $(A : V \rightarrow V)$

    \Def Характеристический многочлен оператора $\mathcal{A} : V \rightarrow V$
    ($\mathcal{A}x = Ax, A$ - матрица в неком базисе)

    $\xi(\lambda) = \det(A - \lambda I)$

    \Nota Матрица $A - \lambda I$:

    $\displaystyle \begin{vmatrix}a_{11} - \lambda & \dots & a_{1n} \\ \vdots & \ddots & \vdots \\ a_{n1} & \dots & a_{nn} - \lambda \end{vmatrix}$

    \Nota Уравнение $\xi(\lambda) = 0$ называется вековым

    \Def Собственным вектором оператора $\mathcal{A}$, отвечающим собственному значению $\lambda$,
    называется $x \neq 0 \ | \ \mathcal{A}x = \lambda x$

    \Def Собственное подпространство оператора $\mathcal{A}$, отвечающее числу $\displaystyle \lambda_i$,

    $\displaystyle U_{\lambda_i} = \Set{x \in V \ | \ \mathcal{A}x = \lambda_i x} \union \Set{0}$

    \Def $\displaystyle \dim U_{\lambda_i} = \beta$ - геометрическая кратность число $\displaystyle \lambda_i$

    \Th $\displaystyle \mathcal{A}x = \lambda x \Longleftrightarrow \det(A - \lambda I) = 0, \quad A : V^n \rightarrow V^n$

    $\Box \Longleftrightarrow |A - \lambda I| = 0 \Longleftrightarrow rang (A - \lambda I) < n \Longleftrightarrow
    \dim Im(A - \lambda I) < n \Longleftrightarrow \dim Ker(A - \lambda I) \geq 1$

    $\exists x \in Ker(A - \lambda I), x \neq 0 \ | \ (A - \lambda I) x = 0 \Longleftrightarrow Ax - \lambda I x = 0 \Longleftrightarrow Ax = \lambda x$

    \Nota По основной теореме алгебры вековое уравнение имеет $n$ корней (не всех из них вещественные).
    В конкретном множестве $\mathcal{K} \ni \lambda$ их может не быть

    \Def Кратность корня $\displaystyle \lambda_i$ называется алгебраической кратностью

    \Th $\displaystyle \lambda_1 \neq \lambda_2 (\mathcal{A}x_1 = \lambda_1 x_1, \mathcal{A}x_2 = \lambda_2 x_2) \Longrightarrow x_1, x_2$ - линейно независимы

    $\Box$ Составим комбинацию: $\displaystyle c_1 x_1 + c_2 x_2 = 0 \quad \Big| \cdot \mathcal{A}$

    $\displaystyle \lambda_1 \neq \lambda_2 \Longrightarrow \lambda_1^2 + \lambda_2^2 \neq 0, \sqsupset \lambda_2 \neq 0$

    $\displaystyle c_1 \mathcal{A} x_1 + c_2 \mathcal{A} x_2 = 0 \Longleftrightarrow c_1 \lambda_1 x_1 + c_2 \lambda_2 x_2 = 0$

    Умножим $\displaystyle c_1 x_1 + c_2 x_2 = 0$ на $\displaystyle \lambda_2$: $\displaystyle c_1 \lambda_2 x_1 + c_2 \lambda_2 x_2 = 0$

    $\displaystyle c_1 \lambda_1 x_1 + c_2 \lambda_2 x_2 - c_1 \lambda_2 x_1 - c_2 \lambda_2 x_2 = 0$

    $\displaystyle c_1 x_1(\lambda_1 - \lambda_2) = 0$

    Так как $\displaystyle \lambda_1 \neq \lambda_2$ по условию, $\displaystyle x_1 \neq 0$ - собственный вектор, поэтому $\displaystyle c_1 = 0$, а комбинация линейно независима

    Если $\displaystyle \lambda_1 = 0, \lambda_2 \neq 0$: $\displaystyle c_2 \lambda_2 x_2 = 0 \Longrightarrow c_2 = 0$

    \Nota Приняв доказательство за базу индукции, можно доказать линейную независимость для $k$-ой системы собственных векторов для попарно различных $k$ чисел $\lambda$

\end{document}




