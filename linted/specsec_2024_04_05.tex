\documentclass[12pt]{article}
\usepackage{preamble}

\pagestyle{fancy}
\fancyhead[LO,LE]{Специальные разделы \\ высшей математики}
\fancyhead[CO,CE]{05.04.2024}
\fancyhead[RO,RE]{Лекции Далевской О. П.}


\begin{document}

    \Mem $(x, y)$ в $\Real$

    $(x, y) = (y, x)$

    Но. $(x, y)$ в комплексном множестве

    $(x, y) = \overline{(y, x)}$

    Важно: линейность по первому аргументу - везде

    $(\lambda x, y) \stackrel{\Real, \mathcal{C}}{=} \lambda (x, y)$

    Но:

    $(x, \lambda y) = \lambda (x, y)$ в $\Real$

    $(x, \lambda y) = \overline{\lambda} (x, y)$ в $\mathcal{C}$

    \DefN{1} Оператор $\displaystyle \mathcal{A}^*$ называется сопряженным для $\mathcal{A} : V \to V$, если

    $\displaystyle (\mathcal{A}x, y) = (x, \mathcal{A}^* y)$

    \DefN{2} $\displaystyle \mathcal{A}^*$ сопряженный для $\mathcal{A}$, если $\displaystyle A^* = A^T$ в любой ортонормированном базисе

    \textbf{Def. 1.} \Longleftrightarrow \textbf{Def. 2.}

    $\displaystyle (\mathcal{A}x, y) \stackrel{\text{на языке матриц}}{=\joinrel=\joinrel=\joinrel=\joinrel=\joinrel=} (AX, Y) = (AX)^T \cdot Y = X^T \cdot A^T \cdot Y$

    $\displaystyle \stackrel{||}{(x, \mathcal{A}^* y)} = X^T \cdot (A^* Y) = (X^T A^*) \cdot Y = X^T \cdot A^T \cdot Y \Longrightarrow A^* = A^T$

    \Lab Очевидно существование $\displaystyle \mathcal{A}^* \ \forall \mathcal{A}$ (определяется в ортонормированном базисе действием $\displaystyle \mathcal{A}^T$)

    Доказать единственность $\displaystyle \mathcal{A}^*$ рассмотреть от противного $\displaystyle (x, \mathcal{A}_1^* y) \neq (x, \mathcal{A}_2^* y)$)

    Свойства:

    1) $\displaystyle \mathcal{I} = \mathcal{I}^* \quad \Box (\mathcal{I}x, y) = (x, y) = (x, \mathcal{I}y) \qed$

    2) $\displaystyle (\mathcal{A} + \mathcal{B})^* = \mathcal{A}^* + \mathcal{B}^*$

    3) $\displaystyle (\lambda \mathcal{A})^* = \lambda \mathcal{A}^*$

    4) $\displaystyle (\mathcal{A}^*)^* = \mathcal{A}$

    5) $\displaystyle (\mathcal{A}\mathcal{B})^* = \mathcal{B}^* \mathcal{A}^*$ (св-во транспонирования матриц)

    или $\displaystyle ((\mathcal{AB})x, y) = (\mathcal{A}(\mathcal{B}x), y) = (\mathcal{B}x, \mathcal{A}^* y) = (x, \mathcal{B}^* \mathcal{A}^* y)$

    6) $\displaystyle \mathcal{A}^*$ - линейный оператор ($\displaystyle \mathcal{A}x = x^\prime, \mathcal{A}y = y^\prime \Longrightarrow \mathcal{A}(\lambda x + \mu y) = \lambda x^\prime + \mu y^\prime$)

    Можно использовать линейные свойства умножения матриц $\displaystyle A^* (\lambda X + \mu Y) = \lambda \mathcal{A}^* X + \mu \mathcal{A}^* Y$

    \textbf{2* Самосопряженный оператор}

    \Def $\mathcal{A}$ называется самосопряженным, если $\displaystyle \mathcal{A} = \mathcal{A}^*$

    Следствие. $\displaystyle A^T = A \Longrightarrow$ матрица $A$ симметричная

    Свойства самосопряженных операторов:

    1) $\displaystyle \mathcal{A} = \mathcal{A}^*, \ \lambda : \ \mathcal{A}x = \lambda x (x \neq 0)$. Тогда, $\lambda \in \Real$

    $\displaystyle \Box (\mathcal{A}x, y) = (\lambda x, y) = \lambda (x, y) \quad (x, \mathcal{A}^* y) = (x, \mathcal{A}y) = (x, \lambda y) \stackrel{\text{ в } \mathcal{C}}{=} \overline{\lambda} (x, y)$

    $(\mathcal{A}x, y) = (x, \mathcal{A}y) \Longrightarrow \lambda (x, y) = \overline{\lambda} (x, y) \Longrightarrow \lambda = \overline{\lambda} \Longrightarrow \lambda \in \Real$

    $\qed$

    2) $\displaystyle \mathcal{A} = \mathcal{A}^*, \ \mathcal{A}x_1 = \lambda_1 x_1, \mathcal{A}x_2 = \lambda_2 x_2$ и $\displaystyle \lambda_1 \neq \lambda_2$

    Тогда $\displaystyle x_1 \perp x_2$

    $\Box$ Хотим доказать, что $\displaystyle (x_1, x_2) = 0$, при том, что $\displaystyle x_{1,2} \neq 0$

    $\displaystyle \lambda_1 (x_1, x_2) = (\lamdba_1 x_1, x_2) = (\mathcal{A} x_1, x_2) = (x_1, \mathcal{A} x_2) = (x_1, \lambda_2 x_2) = (x_1, x_2) \lambda_2$

    Так как $\displaystyle \lambda_1 \neq \lambda_2$, то $\displaystyle (\lambda_1 - \lambda_2) (x_1, x_2) = 0 \Longrightarrow (x_1, x_2) = 0 \qed$

    \Th Лемма. $\displaystyle \mathcal{A} = \mathcal{A}^*$, $e$ - собственный вектор ($\displaystyle l_{\Set{e}}$ - линейная оболочка $e$ - инвариантное подпространство для $\mathcal{A}$)

    $\displaystyle V_1 = \Set{x \in V \ | \ x \perp e}$

    Тогда $\displaystyle V_1$ - инвариантное для $\mathcal{A}$

    $\Box$ Нужно доказать, что $\displaystyle \forall x \in V_1 \ \mathcal{A}x \in V_1$ и так как $\displaystyle x \in V_1 \ | \ x \perp e$, то
    покажем, что $\mathcal{A}x \perp e$

    $(\mathcal{A}x, e) = (x, \mathcal{A}e) = (x, \lambda e) = \lambda (x, e) \stackrel{x \perp e}{=\joinrel=} 0$

    $\qed$

    \Th $\displaystyle \mathcal{A} = \mathcal{A}^*$ ($\displaystyle \mathcal{A} : V^n \to V^n$),
    тогда $\displaystyle \exists e_1, \dots, e_n$ - набор собственных векторов $\mathcal{A}$ и $\displaystyle \Set{e_i}$ - ортонормированный базис

    (другими словами: $\mathcal{A}$ - диагонализируем)

    Наводящие соображения.

    \ExN{1} $A = \begin{pmatrix}1 & 0 & 0 \\ 0 & 1 & 0 \\ 0 & 0 & 1\end{pmatrix} = I$

    $\displaystyle \mathcal{I}x = x = 1 \cdot x, \quad \lambda_{1,2,3} = 1$

    Здесь $\displaystyle U_{\lambda_{1,2,3}} = V^3, \ \Set{\overrightarrow{i}, \overrightarrow{j}, \overrightarrow{k}}$ - базис из собственных векторов, ортонормированный

    \ExN{2} $A = \begin{pmatrix}0 & 0 & 0 \\ 0 & 0 & 0 \\ 0 & 0 & 0\end{pmatrix} = \mathcal{O}$

    $\displaystyle \mathcal{O}x = 0, \quad \lambda_{1,2,3} = 0$

    И здесь $\displaystyle U_{\lambda_{1,2,3}} = V^3$, так как $\displaystyle 0 \in U_\lambda$ и $\displaystyle \forall x \ \mathcal{O}x = 0 \in U_\lambda$

    \ExN{3} Поворот $\displaystyle \Real^2$ на $\displaystyle \frac{\pi}{4}$

    $\displaystyle T = \begin{pmatrix}\frac{1}{\sqrt{2}} & -\frac{1}{\sqrt{2}} \\ \frac{1}{\sqrt{2}} & \frac{1}{\sqrt{2}}\end{pmatrix}$

    $\displaystyle \begin{vmatrix}\frac{1}{\sqrt{2}} - \lambda & -\frac{1}{\sqrt{2}} \\ \frac{1}{\sqrt{2}} & \frac{1}{\sqrt{2}} - \lambda\end{vmatrix} =
    \left(\frac{1}{\sqrt{2}} - \lambda\right)^2 + \frac{1}{2} = 0$ - вещественных корней нет

    $\displaystyle \Box \sqsupset e_1$ - какой-либо собственный вектор $\mathcal{A}$ ...

\end{document}
