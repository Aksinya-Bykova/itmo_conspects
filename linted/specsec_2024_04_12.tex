\documentclass[12pt]{article}
\usepackage{preamble}

\pagestyle{fancy}
\fancyhead[LO,LE]{Специальные разделы \\ высшей математики}
\fancyhead[CO,CE]{12.04.2024}
\fancyhead[RO,RE]{Лекции Далевской О. П.}


\begin{document}
    \Th $\displaystyle \mathcal{A}: V^n \to V^n, \mathcal{A} = \mathcal{A}^* \ \Longrightarrow \ \exists \Set{e_i}^n_{i=1}, e_1$ -
    собственные вектора $\mathcal{A}$ и $\displaystyle \Set{e_i}$ - ортонормированный базис

    $\displaystyle \Box \ \supsqset e_1$ - собственный вектор $\mathcal{A}$

    $\displaystyle e_1$ найдется, если $\mathcal{A}x = \lambda x$ имеет нетривиального решение \ \Longleftrightarrow \
    $\det(\mathcal{A} - \lambda I) = 0$ \ \stackrel{\mathcal{A}\text{ - самосопр.}}{\Longrightarrow} \ $\exists \lambda \in \Real$

    Для вектора $\displaystyle e_1$ строим инвариантное подпространство $\displaystyle V_1 \perp e_1$ (см. лемму), $\displaystyle \dim V_1 = n - 1$

    В подпространстве $\displaystyle V_1$ $\mathcal{A}$ действует как самосопряженный и имеет собственный вектор $\displaystyle e_2 \perp e_1$.
    Для $\displaystyle e_2$ строим $\displaystyle V_2 \perp e_2, e_1$

    Затем, $\displaystyle V_3, V_4, V_5, \dots,$ в котором, найдя $\displaystyle e_i$, ортогональный всем предыдущим

    Составили ортогональный базис из $\displaystyle e_i$, который можно нормировать

    $\Box$

    \Nota Чтобы упорядочить построение базиса, в котором $\displaystyle V_i$ может брать $\displaystyle \max \lambda_i$

    \Nota Из теоремы следует, что самосопряженный оператор диагонализируется: $\displaystyle \Sigma$ алг. крат. $ = n$ (степень уравнения), а $\displaystyle \Sigma$ геом. крат. $\displaystyle = \dim \Set{e_1, \dots, _n} = n$


    Разложение самосопряж. оператора в спектр:

    $\displaystyle x \in V^n \quad \Set{e_i}_{i=1}^n$ - базис из собственных векторов $\mathcal{A}$ (ортонорм.)

    $\displaystyle x = x_1 e_1 + \dots + x_n e_n = (x, e_1) e_1 + \dots + (x, e_n) e_n = \overset{n}{\underset{i = 1}{\Sigma}} (x, e_i) e_i$

    \Def Оператор $\displaystyle P_i x = (x, e_i) e_i$ называется проектором на одномерное пространство, порожденное $\displaystyle e_i$ (линейная оболочка)

    Свойства:

    1) $\displaystyle P_i^2 = P_i$ (более того $\displaystyle P^m_i = P_i$)

    2) $\displaystyle P_i P_j = 0$

    3) $\displaystyle P_i = P_i^* \quad ((P_i x, y) \stackrel{?}{=} (x, P_i y)) \Longleftrightarrow (P_i x, y) = ((x, e_i) e_i, y) = (x, e_i) (e_i, y) = (x, (y, e_i) e_i) = (x, P_i y)$

    Итак, если $\displaystyle \mathcal{A}: V^n \to V^n$ - самосопряженный и $\displaystyle \Set{e_i}$ - ортонормированный базис собственных векторов $\mathcal{A}$, то

    $\displaystyle x = \overset{n}{\underset{i = 1}{\Sigma}} P_i x = \overset{n}{\underset{i = 1}{\Sigma}} (x, e_i) e_i$

    $\displaystyle \mathcal{A} x \stackrel{y = \Sigma (y, e_i) e_i}{=} \overset{n}{\underset{i = 1}{\Sigma}} (\mathcal{A}x, e_i) e_i =
    \overset{n}{\underset{i = 1}{\Sigma}} (x, \mathcal{A}e_i) e_i = \overset{n}{\underset{i = 1}{\Sigma}} (x, \lambda_i e_i) e_i =
    \overset{n}{\underset{i = 1}{\Sigma}} \lambda_i (x, e_i) e_i = \overset{n}{\underset{i = 1}{\Sigma}} \lambda_i P_i x$

    $\displaystyle \Longleftrightarrow \mathcal{A} = \overset{n}{\underset{i = 1}{\Sigma}} \lambda_i P_i$ - спектральное разложение $\mathcal{A}$,
    спектр $\displaystyle = \Set{\lambda_1, \dots, \lambda_n \ | \ \lambda_i \leq \dots \leq \lambda_n}$

    \Ex

    $\displaystyle y = y_1 e_1 + y_2 e_2 = (y, e_1) e_1 + (y, e_2) e_2 = (\mathcal{A}x, e_1) e_1 + (\mathcal{A}x, e_2) e_2 = \lambda_1 x_1 e_1 + \lambda_2 x_2 e_2$

    \section[p2\_9]{2.9. Ортогональный оператор}

    \Mem Орт. оператор $\displaystyle T: V^n \to V^n \overset{def}{\Longleftrightarrow} \forall$ о/н базиса матрица $T$ - ортогональная $\displaystyle T^{-1} = T^T$

    \Nota Иначе, $T$ - ортогональный оператор \Longleftrightarrow $\displaystyle T^{-1} = T^*$ \Longrightarrow $\displaystyle T T^* = I$

    \Def $T$ - ортог. оператор, если $\displaystyle (T_x, T_y) = (x, y)$

    Следствие: $\|Tx\| = \|x\|$, то есть $T$ сохраняет расстояние

    \Nota Ранее в теореме об изменении матрицы $A$ при преобразовании координат $T$ - ортогональный оператор

    Это необязательно, то есть можно переходить в другой произвольный базис (док-во теоремы позволяет)

    Диагонализация самосопряженного оператора:

    Дана матрица $\displaystyle A_f$

    1) Находим $\displaystyle \lambda_1, \dots, \lambda_n$

    2) Находим $\displaystyle e_1, \dots e_n$ - ортогональный базис собственных векторов

    3) Составляем $\displaystyle T = \begin{pmatrix}e_{11} & \dots & e_{1n} \\ \vdots & \ddots & \vdots \\ e_{n1} & \dots & e_{nn}\end{pmatrix}$ - матрица поворота базиса

    4) Находим $\displaystyle T_{e\to f}A_f T_{f\to e} = A_e$ - диагональная

    Таким образом диагонализация самосопряженного $\mathcal{A}$ - это нахождение композиции поворотов и симметрий,
    как приведение пространства к главным направлением

    \clearpage

    \section[p3]{3. Билинейные и квадратичные формы}

    \section[p3\_1]{3.1. Билинейные формы}

    \Def $\displaystyle x, y \in V^n \quad$ Отображение $\displaystyle \mathcal{B}: V^n \to \Real$ (обозн. $\mathcal{B}(x, y)$)
    называется билинейной формой, если выполнены

    1) $\mathcal{B}(\lambda x + \mu y, z) = \lambda \mathcal{B}(x, z) + \mu \mathcal{B}(y, z)$

    2) $\mathcal{B}(x, \lambda y + \mu z) = \lambda \mathcal{B}(x, y) + \mu \mathcal{B}(x, z)$

    \Ex

    1) $\displaystyle \mathcal{B}(x, y) \stackrel{\text{в } E^n_\Real}{=} (x, y)$

    2) $\displaystyle \mathcal{B}(x, y) = P_y x$ - проектор $x$ на $y$

    Матрица Б.Ф.

    \Th $\displaystyle \Set{e_i}_{i=1}^n$ - базис $\displaystyle V_n$, $\displaystyle u, v \in V^n$. Тогда $\displaystyle \mathcal{B}(u, v) =
    \overset{n}{\underset{j = 1}{\Sigma}}\overset{n}{\underset{i = 1}{\Sigma}} b_{ij} u_i v_j$, где $\displaystyle b_{ij} \in \Real$

    $\displaystyle \Box \ \begin{matrix}u = u_1 e_1 + \dots + u_n e_n \\ v = v_1 e_1 + \dots + v_n e_n\end{matrix} \quad
    \mathcal{B}(u, v) = \mathcal{B}(\overset{n}{\underset{i = 1}{\Sigma}} u_i e_i, \overset{n}{\underset{j = 1}{\Sigma}} v_j e_j) =
    \overset{n}{\underset{i = 1}{\Sigma}} u_i \mathcal{B}(e_i, \overset{n}{\underset{j = 1}{\Sigma}} v_j e_j) =
    \overset{n}{\underset{i = 1}{\Sigma}} u_i (\overset{n}{\underset{j = 1}{\Sigma}} v_j \mathcal{B}(e_i, e_j)) = \overset{\text{обозн. } \mathcal{B}(e_i, e_j) = b_{ij}}{=}
    \overset{n}{\underset{i = 1}{\Sigma}} u_i \overset{n}{\underset{j = 1}{\Sigma}} v_j b_{ij} = \overset{n}{\underset{i = 1}{\Sigma}} \overset{n}{\underset{j = 1}{\Sigma}} u_i v_j b_{ij}$

    $\Box$

    \Nota Составим матрицу из $\displaystyle \mathcal{B}(e_i, e_j)$

    $\displaystyle B = \begin{pmatrix}b_{11} & \dots & b_{1n} \\ \vdots & \ddots & \vdots \\ b_{n1} & \dots & b_{nn}\end{pmatrix}$

    \Def Если

    1) $\mathcal{B}(u, v) = \mathcal{B}(v, u)$, то $\mathcal{B}$ - симметричная

    2) $\mathcal{B}(u, v) = -\mathcal{B}(v, u)$, то $\mathcal{B}$ - антисимметричная

    3) $\mathcal{B}(u, v) = \overline{\mathcal{B}(v, u)}$, то $\mathcal{B}$ - кососимметричная (в $\mathcal{C}$)

    \Def $rang \mathcal{B}(u, v) \stackrel{def}{=} rang B$

    \Nota

    1) $\mathcal{B}$ называется невырожденной, если $rang \mathcal{B} = n$

    2) $\displaystyle rang \mathcal{B}_e = rang \mathcal{B}_{e^\prime} $ ($\displaystyle e, e^\prime$ - различные базисы $\displaystyle V^n$), то есть $rang \mathcal{B}$ инвариантно относительно преобразования $\displaystyle e \to e^\prime$

    \Ex $\mathcal{B}(u, v) \stackrel{\text{ск. пр.}}{=} (u, v)$

    $\displaystyle \begin{matrix}u = u_1 e_1 + u_2 e_2 \\ v = v_1 e_1 + v_2 e_2\end{matrix}$, тогда $\displaystyle \mathcal{B}(e_i, e_j) = \stackrel{\text{об}}{=} b_{ij} = (e_i, e_j)$

    Таким образом, $\displaystyle B = \begin{pmatrix}(e_1, e_1) & (e_1, e_2) \\ (e_2, e_1) & (e_2, e_2)\end{pmatrix}$ - матрица Грама

    \Ex $\begin{matrix}u(t) = 1 + 3t \\ v(t) = 2 - t\end{matrix}$, $\displaystyle \Set{e_i} = (1, t)$, $\displaystyle \mathcal{B}(u, v) = (u, v) = \int_{-1}^1 uv dt$

    Тогда, $\displaystyle B = \begin{pmatrix}\int_{-1}^1 dt & \int_{-1}^1 t dt \\ \int_{-1}^1 t dt & \int_{-1}^1 t^2 dt\end{pmatrix} = \begin{pmatrix}2 & 0 \\ 0 & \frac{2}{3}\end{pmatrix}$

\end{document}

