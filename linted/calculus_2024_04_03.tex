\documentclass[12pt]{article}
\usepackage{preamble}

\pagestyle{fancy}
\fancyhead[LO,LE]{Математический анализ}
\fancyhead[CO,CE]{03.04.2024}
\fancyhead[RO,RE]{Лекции Далевской О. П.}


\begin{document}
    \textbf{5.3 Двойной и тройной интегралы}

    \vspace{3mm}
\textit{Nota}. Дадим строгое определение

    \begin{multicols}{2}
        \vspace{3mm}
\textit{Def}. $z = z(x, y) \quad z : D \subset \Real^2 \rightarrow \Real$

        1) Дробление на $[x_{i-1}, x_i]$ длиной $\Delta x$

        2) Выбор средней точки $M_i(\xi_i, \eta_i)$, по значению $z(M_i)$ строим элементарный параллелепипед объемом
        $\nu_i = z(M_i) \Delta x_i \Delta y_i \approx V_{\text{малого цилиндра}}$

        3) Интеграл суммы $\displaystyle v_i = \Sigma_{i=1}^n \nu_i = \Sigma z(M_i) \Delta x_i \Delta y_i$

        4) Если $\exists \lim v_n \in \Real$, не зависящий от ранга, типа дробления и т.д. при $n \rightarrow \infty$ и
        $\tau = \max \Delta x_i, \Delta y_i \to 0$, то $\displaystyle \lim_{n\to\infty} v_n \stackrel{def}{=} \iint_D z(x, y) dx dy$ - двойной интеграл от $z(x, y)$ на области $D$

        \vspace{3mm}
\textit{Mem}. $\displaystyle \int_a^b f(x) dx$

        $f(x) : [a, b] \rightarrow \Real^+$

        1) Дробление на элементы $P_i$ прямыми $x = const, y = const$, $S_{P_i} = \Delta x_i \Delta y_i$ (дали $dx$, $dy$)

        2) Выбор $\xi_i \in [x_{i-1}, x_i]$, площадь элементарных прямоугольников $f(\xi_i)\Delta x_i \approx S_{\text{полоски}}$

        3) Интеграл суммы $\displaystyle \sigma_n = \Sigma_{i=1}^n f(\xi_i) \Delta x_i$

        4) $\displaystyle \lim_{n\to\infty} \sigma_n = \int^b_a f(x) dx$
    \end{multicols}

    \vspace{3mm}
\textit{Nota}: Об области $D$

    В простейшем случае рассматривают выпуклую, односвязную $\Real^2$-область

    \vspace{5mm}
    \textbf{
    a) Выпуклость:}

    $\exists M_1, M_2 \in D \ | \ \overline{M_1 M_2} \notin D$ - не выпуклая

    $\forall M_1, M_2 \in D \ | \ \overline{M_1 M_2} \in D$ - выпуклая

    \vspace{5mm}
    \textbf{
    б) Связность:}

    $D = D^\prime \union D^{\prime\prime}$ - не связная: $\exists M_1, M_2 \in D \ | \ \overset{\LARGE\smile}{M_1 M_2} \notin D$

    $D$ - связная: $\forall M_1, M_2 \in D \ | \ \overset{\LARGE\smile}{M_1 M_2} \in D$

    Обычно область - открытая, дальше будем рассматривать в том числе области с границей.

    Добавим к определению $\displaystyle \iint_{\partial D \text{ - граница } D} z(x, y) dx dy$

    Геометрический смысл: В определении при $z(x, y) \geq 0$ интегральная сумма $\displaystyle v_n = \Sigma_{i=1}^n \nu_i$ была суммой объемов элементарных параллелепипедов и приближала объем подповерхности

    Тогда $\displaystyle \iint_D z(x, y) dx dy \stackrel{z \geq 0}{=} V_{\text{цилиндра с осн. } D}$, а при $z = 1$ $\displaystyle \iint_D dx dy = S_D$

    Вычисление: По геометрическому смыслу - найти $\displaystyle \iint_D z(x, y) dx dy$ значит найти объем подповерхности

    Можно найти $\displaystyle S(x) = \int^{y_2(x)}_{y_1(x)} z(x = c, y) dy$ - площадь поперечного сечения

    Найдем $V$ как объем тела с известными площадями сечений

    $\displaystyle V = \int^b_a S(x) dx = \int_a^b \left(\int^{y_2(x)}_{y_1(x)} z(x = c, y) dy\right) dx$

    \vspace{3mm}
\textit{Nota}. Кратный

    Если найдена первообразная для $z(x = c, y)$ (обозн. $F(x, y(x))$), то по формуле N-L:

    $\displaystyle \int^{y_2(x)}_{y_1(x)} z(x = c, y) dy = F(x, y(x)) \Big|^{y_2(x)}_{y_1(x)} = F(x, y_2(x)) - F(x, y_1(x))$

    Тогда $\displaystyle \int^b_a \stackrel{\varphi(x)}{\overgroup{(F(x, y_2) - F(x, y_1))}} dx$ - обычный определенный интеграл

    Пределы интегрирования во внутреннем интеграле - функции, во внешнем - точки

    ? Можно вычислить V, рассекая тело сечениями $y = const$. Верно ли, что $\displaystyle \int_a^b \left(\int_{y_1(x)}^{y_2(x)} z(x, y) dy\right) dx = \int_\alpha^\beta \left(\int_{x_1(y)}^{x_2(y)} z(x, y) dx\right) dy$?

    Верно, $V$ не зависит от порядка сечения

    Таким образом, двойной интеграл $\displaystyle \iint_D z(x, y) dxdy = \int_a^b \int_{y_1}^{y_2} z(x, y) dydx = \int_\alpha^\beta \int_{x_1}^{x_2} z(x, y) dxdy$

    Но при другом порядке интегрирования область $D$ может оказаться неправильной

    \vspace{3mm}
\textit{Def}. При проходе области $D$ в направлении $Oy \uparrow$ граница области (верхняя) меняет аналитическое задание. Такая область называется направильной в направлении $Oy$

    Выгодно выбирать правильное направление, чтобы не делить интеграл по аддитивности

    \vspace{3mm}
\textit{Ex}. $\displaystyle \iint_D xy dx dy$, $D : x^2 + y^2 \leq 1$

    $\displaystyle \iint_D xy dx dy = \int_{-1}^1 (\int_{y_1 = -\sqrt{1-x^2}}^{y_2 = \sqrt{1-x^2}} xy dy) dx = \int_{-1}^1 (\frac{x}{2} y^2 \Big|_{y_1 = -\sqrt{1-x^2}}^{y_2 = \sqrt{1-x^2}}) dx =
    \int_{-1}^1 (\frac{x}{2} ((1 - x^2) - (1 - x^2)) dx = 0$

    \vspace{3mm}
\textit{Def}. Тройной интеграл

    $T \subset \Real^3 \rightarrow \Real$

    1) дробление на элементы объема $dv = dxdydz$

    2) Вычисление среднего содержания $u(x, y, z)$ в $dv$: $u(\xi_i, \eta_i, \zeta_i) dv$

    3) Интегральная сумма $\displaystyle \sigma_n = \Sigma u(M_i) dv$

    4) $\displaystyle \lim_{n \to \infty, \tau = \max dv \to 0} \stackrel{def}{=} \iiint_T u(x, y, z) dxdydz$

    Геометрический смысл. Только при $u = 1$ $\displaystyle \iiint_T dxdydz = V_T$

    Физический смысл. $u(x, y, z)$ - плотность в каждой точке $T$

    $\displaystyle \iiint_T u(x, y, z) dxdydz = m_T$ - масса

    Вычисление. $\displaystyle \iiint_T u(x, y, z) dxdydz \stackrel{\text{кратный}}{=} \int^b_a \int_{y_1(x)}^{y_2(x)} \int_{z_1(x, y)}^{z_2(x, y)} u(x, y, z) dz dy dx$

\end{document}
