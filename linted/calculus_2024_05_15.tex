\documentclass[12pt]{article}
\usepackage{preamble}

\pagestyle{fancy}
\fancyhead[LO,LE]{Математический анализ}
\fancyhead[CO,CE]{15.05.2024}
\fancyhead[RO,RE]{Лекции Далевской О. П.}


\begin{document}

    Разберем пример поверхностного интеграла:

    \Exs $\displaystyle S_1:\ x^2 + y^2 = 1, \quad S_2: z = 0, \quad S_3: z = 1$

    $\displaystyle S = \bigunion_{i = 1}^3 S_i$ - цилиндр

    $\overrightarrow{F} = (P, Q, R) = (x, y, z)$

    $\displaystyle \iint_{S_{\text{внешн.}}} x dy dz + y dx dz + z dx dy = \iint_{S_1} + \iint_{S_2} + \iint_{S_3}$

    Так как проекции $\displaystyle S_2$ на $Oxz$ и $Oyz$ - отрезки, то $dxdz = 0$, $dydz = 0$

    $\displaystyle \iint_{S_2} xdydz + ydxdz + zdxdy = \iint_{S_2} zdxdy = 0$

    $\displaystyle \iint_{S_3} zdxdy \stackrel{z |_{S_3} = 1}{=} \iint_{S_3} dxdy \stackrel{\text{с "+", так как }\overrightaarrow{n_3} \uparrow\uparrow Oz}{=} \iint_{D_{xy}} dxdy = \pi$

    $\displaystyle \iint_{S_1} xdydz + ydxdz = \iint_{D^+_{yz}: x = \sqrt{1 - y^2}} xdydz + (-\iint_{D^-_{yz}: x = -\sqrt{1 - y^2}} xdydz) + \iint_{D^+_{xz}} ydxdz + (-\iint_{D^-_{xz}} ydxdz) = \dots$


    \section{5.7. Связь поверхностных интегралов с другими}

    \Th Гаусса-Остроградского

    $\displaystyle S_1\ : \ z = z_1(x, y),\ S_3\ :\ z = z_3(x, y),\ S_2\ : \ f(x, y) = 0$ (проекция на $Oxy$ - кривая)

    $\displaystyle S = \bigunion_{i = 1}^3 S_i$ - замкнута! и ограничивает тело $T$

    $P = P(x, y, z), Q = Q(x, y, z), R = R(x, y, z)$ - непр. дифф., действуют в области $\Omega \supset T$

    Тогда $\displaystyle \oiint_{S_{\text{внешн.}}} Pdydz + Qdxdz + Rdxdy = \iiint_T (\frac{\partial P}{\partial x} + \frac{\partial Q}{\partial y} + \frac{\partial R}{\partial z}) dxdydz$

    \Mem Формула Грина

    $\displaystyle \oint_K Pdx + Qdy = \iint_{D_{xy}} (\frac{\partial Q}{\partial x} - \frac{\partial Q}{\partial y}) dxdy$

    $\Box$

    Вычислим почленно $\displaystyle \iiint_T (\frac{\partial P}{\partial x} + \frac{\partial Q}{\partial y} + \frac{\partial R}{\partial z}) dv$

    $\displaystyle \iiint_T (\frac{\partial R(x, y, z)}{\partial z} dz) dxdy = \iint_{D_{xy}} R(x, y, z) \Big|_{z = z_1 (x, y)}^{z = z_3(x, y)} dxdy = \iint_{D_{xy}} (R(x, y, z_3(x, y)) - R(x, y, z_1(x, y))) dxdy =
    \underset{\text{двойной}}{\iint_{D_{xy}} R(x, y, z_3) dxdy} - \iint_{D_{xy}} R(x, y, z_1(x, y)) dxdy = \underset{\text{поверхностный}}{\iint_{S_3} R(x, y, z) dxdy} + \iint_{S_1} R(x, y, z) dxdy +
    \underset{ =\ 0,\ dxdy\ |_{S_2} = 0}{\iint_{S_2} R(x, y, z) dxdy} = \iint_{S_{\text{внешн.}}} Rdxdy$

    \vspace{4mm}

    Аналогично остальные члены:

    $\displaystyle \iiint_T \frac{\partial Q}{\partial y} dxdydz = \iint_{S_{\text{внешн.}}} Qdxdz, \iiint_T \frac{\partial P}{\partial y} dxdydz = \iint_{S_{\text{внешн.}}} Pdxdz$

    $\Box$

    \Nota Если $\displaystyle \iint_{S_{\text{внутр}}}$, то $\displaystyle \iint_S = - \iiint_T$

    \Nota С учетом связи поверхностных интегралов $\displaystyle \iiint_T (\frac{\partial P}{\partial x} + \frac{\partial Q}{\partial y} + \frac{\partial R}{\partial z}) dv =
    \iint_S (P\cos\alpha + Q\cos\beta + R\cos\gamma) dv$

    \Th Стокса

    Пусть $S : z = z(x, y)$ - незамкнутая поверхность, $L$ - контур, на которую она опирается

    $\displaystyle \text{пр}_{Oxy} L = K_{xy}, \quad \text{пр}_{Oxy} S = D_{xy}$

    В области $\Omega \supset S$ действуют функции $P, Q, R$ - непр. дифф.

    Тогда $\displaystyle \oint_{L^+} Pdx + Qdy + Rdz = \iint_{S^+} ((\frac{\partial R}{\partial y} - \frac{\partial Q}{\partial z})\cos\alpha +
    (\frac{\partial P}{\partial z} - \frac{\partial R}{\partial x})\cos\alpha + (\frac{\partial Q}{\partial x} - \frac{\partial P}{\partial y})\cos\gamma) d\sigma$

    $\Box$

    Найдем слагаемое $\displaystyle \oint_L P(x, y, z) dx \stackrel{\text{на } L\ : \ z = z(x, y)}{=\joinrel=\joinrel=\joinrel=}
    \oint_{K^+_{xy}} \tilde{P}(x, y, z(x, y)) dx = \oint_{K_{xy}} \tilde{P}dx + \tilde{Q}dy =
    \iint_{D_{xy}} (\frac{\partial \tilde{Q}}{\partial x} - \frac{\partial \tilde{P}}{\partial y}) dxdy =
    -\iint_{D_{xy}} \frac{\partial \tilde{P}(x, y)}{\partial y} dxdy =
    -\iint_{S^+} \frac{\partial P(x, y, z)}{\partial y} dxdy =
    -\iint_{S^+} (\frac{\partial P}{\partial y} + \frac{\partial P}{\partial z} \frac{\partial z}{\partial y}) dxdy =
    -\iint_{S^+} (\frac{\partial P}{\partial y}\cos\gamma + \frac{\partial P}{\partial z} (-\cos\beta)) d\sigma$

    $\displaystyle \overrightarrow{n} = (\frac{-\frac{\partial z}{\partial x}}{\sqrt{1 + z_x^{\prime 2} + z_y^{\prime 2}}})$

    $\displaystyle \cos\gamma = \frac{1}{\sqrt{1 + z_x^{\prime 2} + z_y^{\prime 2}}}$

    Аналогично $\displaystyle \oint_L Qdy = \iint_{S^+} (\frac{\partial Q}{\partial x}\cos\beta - \frac{\partial Q}{\partial z}\cos\alpha) d\sigma,
    \oint_L Rdz = \iint_{S^+} (\frac{\partial R}{\partial y}\cos\alpha - \frac{\partial R}{\partial x}\cos\beta) d\sigma$

    Остается сложить интегралы

    $\Box$

    \ExN{1} $(P, Q, R) = (x, y, z)$

    В \Exs пункте 5.6. (вычисление поверхностного):

    $\displaystyle \iint_{S_{\text{внешн}} \text{ - замкнута!}} xdydz + ydxdz + zdxdy = \iiint_T (\frac{\partial x}{\partial x} + \frac{\partial y}{\partial y} + \frac{\partial z}{\partial z}) dv = 3V_{\text{цил.}}$

    \ExN{2} Те же $P, Q, R$

    $\displaystyle \oint_L Pdx + Qdy + Rdz = \iint_{S} (\overset{= 0}{\overgroup{(\frac{\partial z}{\partial y} - \frac{\partial y}{\partial z})}} \cos\alpha + 0 + 0) d\sigma$

    \clearpage


    \section{6. Теория поля}


    \section{6.1. Определения}

    \DefN{1} $\displaystyle \Omega \supset \Real^n \quad$ Функция $u \ : \ \Omega \to \Real$ называется скалярным полем в $\Omega$

    \DefN{2} Функция $\displaystyle \overrightarrow{F} = (F_1(\overrightarrow{x}), \dots, F_n(\overrightarrow{x})) : \Omega \to \Real^n$ называется векторным полем

    \Nota Далее будем рассматривать функции в $\displaystyle \Real^3$, то есть $u = u(x, y, z)$ и $\overrightarrow{F} = (P(x, y, z), Q(x, y, z), R(x, y, z))$

    \Nota Функции $u$ и $\overrightarrow{F}$ могут зависеть от вренмени $t$. Тогда эти поля называются нестационарными. В противном случае стационарными


    \section{6.2. Геометрические характеристики полей}

    $u = u(x, y, z)$: $l$ - линии уровня $u = const$

    $\overrightarrow{F} = (P, Q, R)$: $w$ - векторная линия, в каждой точке $w$ вектор $\overrightarrow{F}$ - касательная к $w$

    \underline{Векторная трубка} - совокупность непересекающихся векторных линий

    \Nota Отыскание векторных линий

    Возьмем $\overrightarrow{\tau}$ - элементарный касательный вектор, $\overrightarrow{\tau} = (dx, dy, dz)$

    Определение векторной линии: $\displaystyle \overrightarrow{\tau} || \overrightarrow{F} \quad \frac{dx}{P} = \frac{dy}{Q} = \frac{dz}{R}$ - система ДУ

    \Ex $\displaystyle \overrightarrow{F} = y \overrightarrow{i} - x \overrightarrow{j}, M_0 (1, 0)$ - ищем векторную линию $\displaystyle w \ni M_0$

    Задача Коши:

    $\displaystyle \begin{cases}
         \frac{dx}{y} = -\frac{dy}{x} \\
         y(1) = 0
    \end{cases} \Longleftrightarrow \begin{cases}
                                        xdx = -ydy \\
                                        y(1) = 0
    \end{cases} \Longleftrightarrow \begin{cases}
                                        x^2 = -y^2 + C \\
                                        y(1) = 0 \Longrightarrow C = +1
    \end{cases} \Longleftrightarrow x^2 + y^2 = 1 $


    \section{6.3. Дифференциальные характеристики}

    \Mem $\displaystyle \overrightarrow{\triangledown}u = \overrightarrow{grad}u = (\frac{\partial u}{\partial x}; \frac{\partail u}{\partial y}; \frac{\partial u}{\partial z})$ - градиент скалярного поля

    $\displaystyle \overrightarrow{\triangledown} = (\frac{\partial}{\partial x}; \frac{\partail}{\partial y}; \frac{\partial}{\partial z})$ - набла-оператор

    \Nota Для $\overrightarrow{\triangledown}$ определены действия:

    $\displaystyle \overrightarrow{\triangledown} \cdot \overrightarrow{a} = \frac{\partial a_1}{\partial x} + \frac{\partial a_2}{\partial y} + \frac{\partial a_3}{\partial z}$

    $\displaystyle \overrightarrow{\triangledown} \times \overrightarrow{a} =
    \begin{vmatrix}
        \overrightarrow{i}          & \overrightarrow{j}          & \overrightarrow{j}          \\
        \frac{\partial}{\partial x} & \frac{\partial}{\partial y} & \frac{\partial}{\partial z} \\
        a_1                         & a_2                         & a_3
    \end{vmatrix}$

    Причем $\displaystyle \overrightarrow{\triangledown} \cdot \overrightarrow{\triangledown} = \frac{\partial^2}{\partial x^2} + \frac{\partial^2}{\partial y^2} + \frac{\partial^2}{\partial z^2} = \triangle$ - лапласиан

    $\overrightarrow{\triangledown} \times \overrightarrow{\triangledown} = 0$

    \Nota $\displaystyle \triangle u = \underset{\text{часть волнового уравнения матфизики}}{\frac{\partial^2 u}{\partial x^2} + \frac{\partial^2 u}{\partial y^2} + \frac{\partial^2 u}{\partial z^2}} = 0$ - уравнение, определяющее гармоническую функцию $u(x, y, z)$, уравнение Лапласа

    \DefN{1} Дивергенция поля (\textit{divergence} - расхождение)

    $div \overrightarrow{F} \stackrel{def}{=} \overrightarrow{\triangledown} \cdot \overrightarrow{F}$

    \DefN{2} Вихрь (ротор) поля

    $rot \overrightarrow{F} \stackrel{def} \overrightarrow{\triangledown} \times \overrightarrow{F}$

    \DefN{3} Если $rot \overrightarrow{F} = 0$, то $\overrightarrow{F}$ называется безвихревым полем

    \DefN{4} Если $div \overrightarrow{F} = 0$, то $\overrightarrow{F}$ называется соленоидальным

    \Nota Безвихревое поле имеет незамкнутые векторные линии, а вихревое - замкнутые

    \ThN{1} Свойство безвихревого поля

    $rot \overrightarrow{F} = 0 \Longleftrightarrow \exists u(x, y, z) \ | \ \overrightarrow{\triangledown}u = \overrightarrow{F}$

    $\Box$ \fbox{\Longrightarrow}

    $\displaystyle rot \overrightarrow{F} =
    \begin{vmatrix}
        \overrightarrow{i}          & \overrightarrow{j}          & \overrightarrow{j}          \\
        \frac{\partial}{\partial x} & \frac{\partial}{\partial y} & \frac{\partial}{\partial z} \\
        P & Q & R
    \end{vmatrix} = (\frac{\partial R}{\partial y} - \frac{\partial Q}{\partial z})\overrightarrow{i} + (\frac{\partial P}{\partial z} - \frac{\partial R}{\partial x})\overrightarrow{j} + (\frac{\partial Q}{\partial x} - \frac{\partial P}{\partial y})\overrightarrow{k} = 0$

    $\displaystyle \Longleftrightarrow
    \begin{cases}
        \frac{\partial R}{\partial y} = \frac{\partial Q}{\partial z} \\
        \frac{\partial P}{\partial z} = \frac{\partial R}{\partial x} \\
        \frac{\partial Q}{\partial x} = \frac{\partial P}{\partial y}
    \end{cases}$

    Рассмотрим $\displaystyle u = u(x, y, z) \ | \ \frac{\partial u}{\partial x} = P, \frac{\partial u}{\partial y} = Q, \frac{\partial u}{\partial z} = R$ - удовлетворяет системе равенств

    $\displaystyle \overrightarrow{F} = (P, Q, R) = (\frac{\partial u}{\partial x}, \frac{\partial u}{\partial y}, \frac{\partial u}{\partial z}) = \overrightarrow{\triangledown} u$

    \fbox{\Longleftarrow} $\overrightarrow{F} = \overrightarrow{\triangledown}u$ - дана

    $rot \overrightarrow{F} = \overrightarrow{\triangledown} \times \overrightarrow{F} = \overrightarrow{\triangledown} \times (\overrightarrow{\triangledown} u) = (\overrightarrow{\triangledown} \times \overrightarrow{\triangledown}) u = 0$

    $\Box$

    \Nota Доказали, что если векторное поле является градиентом какого-то скалярного, то его вихрь равен нулю: $rot \overrightarrow{grad} u = 0$

    \Def $\overrightarrow{F} = \overrightarrow{\triangledown} u \quad$ Поле $u(x, y, z)$ называется потенциалом поля $\overrightarrow{F}$

    Таким образом, доказано, что безвихревое поле потенциально

    \ThN{2} Свойство соленоидального поля

    $div (rot \overrightarrow{F}) = 0$

    $\Box$

    $div (rot \overrightarrow{F}) = div \overrightarrow{a} = \overrightarrow{\triangledown} \overrightarrow{a} = \overrightarrow{\triangledown} (\overrightarrow{\triangledown} \times \overrightarrow{F}) = (\overrightarrow{\triangledown} \times \overrightarrow{\triangledown}) \cdot \overrightarrow{F} = 0$

    $\Box$

    \section{6.4. Интегральные характеристики. Теоремы теории поля}

    \Mem 1) Поток поля $\displaystyle \overrightarrow{F}: \Pi = \iint_S \overrightarrow{F}d\overrightarrow{sigma}$

    \Def 2) Циркуляция поля $\displaystyle \overrightarrow{F}: \Gamma = \oint_L Pdx + Qdy + Rdz$

    \Nota Запишем \Ths-мы на векторном языке

    1* \textbf{Гаусса-Остроградского}

    $\displaystyle \iint_S Pdydz + Qdxdz + Rdxdy = \iiint_T (\frac{\partial P}{\partial x} + \frac{\partial Q}{\partial y} + \frac{\partial R}{\partial z}) dxdydz$

    $\displaystyle \iint_S (P, Q, R) (dydx, dxdz, dxdy) = \iint_S (P, Q, R) (\cos\alpha d\sigma, \cos\beta d\sigma, \cos\gamma d\sigma) =
    \iint_S \overrightarrow{F} \overrightarrow{n} d\sigma = \iint_S \overrightarrow{F} d\overrightarrow{\sigma}$

    $\displaystyle \iiint_T (\frac{\partial P}{\partial x} + \frac{\partial Q}{\partial y} + \frac{\partial R}{\partial z}) dxdydz = \iiint_T (\overrightarrow{\triangledown} \overrightarrow{F}) = \iiint_T div \overrightarrow{F}$

    \fbox{$\displaystyle \iint_S \overrightarrow{F} d\overrightarrow{\sigma} = \iiint_T div \overrightarrow{F}$}

    \vspace{5mm}

    2* \textbf{Стокса}

    $Pdx + Qdy + Rdz = \overrightarrow{F}d\overrightarrow{l}$

    $\displaystyle \oint_L \overrightarrow{F}d\overrightarrow{l} = \iint_S rot \overrightarrow{F} \overrightarrow{n} d\sigma = \iint_S rot \overrightarrow{F} d\overrightarrow{\sigma}$

    \vspace{5mm}

    3* \textbf{\Th о потенциале}

    $\displaystyle \forall L \ \oint_L \overrightarrow{F}d\overrightarrow{l} = 0 \Longleftrightarrow rot \overrightarrow{F} = 0 \Longleftrightarrow \exists u(x, y, z) \ | \ \overrightarrow{\triangledown} = \overrightarrow{F}$

    (см. \Ths интеграла НЗП)

    \Ex $\overrightarrow{F} = x\overrightarrow{i} + xy \overrightarrow{j}, L: x = y, x = -y, x = 1$

    По формуле Грина (Стокса) $\displaystyle \oint_L \overrightarrow{F} d\overrightarrow{l} = \iint_{D} (\frac{\partial Q}{\partial x} - \frac{\partial P}{\partial y}) dxdy =
    \iint_D y dxdy \quad rot \overrightarrow{F} \neq 0$

    $\displaystyle \oint_L xdx + xydy = \int_{L_1} + \int_{L_2} + \int_{L_3} = \int_0^1 (x + x^2) dx + \int_{-1}^1 y dy - \int_0^1 (x + x^2) dx = \int_{-1}^1 y dy = 0$

    \section{6.5. Механический смысл}

    1* Дивергенция

    Гаусс-Остроградский: $\displaystyle \iiint_T div \overrightarrow{F} dv = \Pi$

    \Ths о среднем: $\displaystyle \exists M_1 \in T \ | \ \iiint_T div \overrightarrow{F} dv = div \overrightarrow{F} \Big|_{M_1} \cdot V_T = \Pi$

    $\displaystyle div \overrightarrow{F} \Big|_{M_1} = \frac{\Pi}{V_T}$, точка $\displaystyle M_0, S$ и $T$ выбраны произвольно

    $\displaystyle \letsymbol V_T \to 0$, тогда $\displaystyle div \overrightarrow{F} \Big|_{M_1 \to M_0} = \lim_{V_T \to 0} \frac{\Pi}{V_T}$ - поток через границу бесконечно малого объема с центром $\displaystyle M_0$, отнесенный к $\displaystyle V_T$ - мощность источника в $\displaystyle M_0$

    Таким образом, дивергенция поля - мощность источников

    \Nota Смысл утверждения $div (rot \overrightarrow F) = 0$ - поле вихря свободно от источников

    \Nota Утверждение $rot (\overrightarrow{grad} u) = 0$ - поле потенциалов свободно от вихрей

    \vspace{5mm}

    2* Ротор

    Стокс $\displaystyle \iint_S rot \overrightarrow{F} d\overrightarrow{\sigma} = \Gamma$

    \Ths о среднем: $\displaystyle \exists M_1 : \iint_S rot \overrightarrow{F} d\overrightarrow{\sigma} = rot \overrightarrow{F} \Big|_{M_1} \cdot S = \Gamma$

    $\displaystyle rot \overrightarrow{F} \Big|_{M_1} = \frac{\Gamma}{S}$, будем стягивать $S$ к точке $\displaystyle M_0$ \Longrightarrow $\displaystyle rot \overrightarrow{F} \Big|_{M_0} = \lim_{S \to 0} \frac{\Gamma}{S}$ - циркуляция по б.м. контуру с центром $\displaystyle M_0$

\end{document}




