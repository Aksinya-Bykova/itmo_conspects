\documentclass[12pt]{article}
\usepackage{preamble}

\pagestyle{fancy}
\fancyhead[LO,LE]{$\mathcal{D}$искретная математика}
\fancyhead[RO,RE]{Лекции Чухарева К. И.}

\renewcommand{\thesection}{}

\begin{document}

    \tableofcontents
    \clearpage

    
    \section{7. Комбинаторика}

    \textbf{Базовые понятия:}

    \begin{itemize}
        \item \textbf{Алфавит} (Alphabet) $\Sigma$ (или $X$, \Exs $X = \Set{a, b, c}$) - множество символов в нашей системе

        \vspace{5mm}

        \item \textbf{Диапазон} (Range) $[n] = \Set{1, \dots, n}$ - конечное множество последовательных натуральных чисел

        \vspace{5mm}

        \item \textbf{Расстановка} (Ordered arrangement) - последовательность каких-либо элементов (тоже самое, что кортеж),
        \Exs $x = (a, b, c, d, b, b, c) \quad |x| = n$

        Расстановку можно представить как функцию $f : \underset{\text{domain}}{\undergroup{[n]}} \to \underset{\text{codomain}}{\undergroup{\Sigma}}$, которая по порядковому номеру выдает символ

        $ran f = \Set{c \in \Sigma \ | \ \exists i \in [n]\ :\ f(i) = c}$

        \vspace{5mm}

        \item \textbf{Перестановка} (Permutation) - $\pi : [n] \to \Sigma$, где $n = |\Sigma|$

        Расстановка $\pi$ - биекция между $[n]$ и $\Sigma$

        \Ex $\pi = \mathtt{2713546}$

        \vspace{3mm}

        \begin{tabular}{l|ccccccc}
            i      & 1          & 2          & 3          & 4          & 5          & 6          & 7          \\
            \hline
            \pi(i) & \mathtt{2} & \mathtt{7} & \mathtt{1} & \mathtt{3} & \mathtt{5} & \mathtt{4} & \mathtt{6}
        \end{tabular}

        \vspace{5mm}

        \underline{Одна из задач комбинаторики} - посчитать количество различных расстановок или перестановок при заданных $n$ и $\Sigma$}

        \vspace{5mm}

        \item \textbf{$k$-перестановка} (k-permutation) - расстановка из $k$ различных элементов из $\Sigma$

        \Ex $\underset{\text{5-perm из } \Sigma = [7]}{\undergroup{|31475|}} = 5$

        $k$-перестановка - инъекция $\pi : [k] \to \Sigma$ ($k \leq n = |\Sigma|$)

        \vspace{5mm}

        \item $P(n, k)$ - множество всех $k$-перестановок алфавита $\Sigma = [n]$ (если исходный алфавит не состоит из чисел, то мы можем сделать биекцию между ним и $[n]$)

        $P(n, k) = \Set{f \ | \ f : [k] \to [n]}$

        Чаще интересует не само множество, а его размер, поэтому под обозначением $P(n, k)$ подразумевается $|P(n, k)|$

        \vspace{5mm}

        \item $\displaystyle S_n = P_n = P(n, n)$ - множество всех перестановок. Также чаще всего нас будет интересовать не множество, а его размер

        $\displaystyle |S_n| = n!$ - всего существует $n!$ перестановок

        $\displaystyle |P(n, k)| = n \cdot (n - 1) \cdot (n - 2) \cdot \dots \cdot (n - k + 1) = \frac{n!}{(n - k)!}$

        \vspace{5mm}

        \item \textbf{Циклические $k$-перестановки} (Circular $k$-permutations)

        $\displaystyle \pi_1, \pi_2 \in P(n, k)$ - циклические эквивалентны тогда и только тогда:

        $\displaystyle \exists s \ | \ \forall i \ \pi_1((i + s) \% k) = \pi_2(i)$


        \Ex $\displaystyle \pi_1 = \mathtt{76123}, \pi_2 = \mathtt{12376}$

        \begin{tikzpicture}
        \node[circle, draw=black!60, thick, minimum size=0.5cm] (s00) {\mathtt{7}};
        \node[below=1cm of s00] (pi1) {$\displaystyle \pi_1$};
        \node[below left=0.66cm and 1cm of s00] (s01) {\mathtt{6}};
        \node[below right=1cm and 0.13cm of s01] (s02) {\mathtt{1}};
        \node[below right=0.66cm and 1cm of s00] (s04) {\mathtt{3}};
        \node[below left=1cm and 0.13cm of s04] (s03) {\mathtt{2}};
        \path[->]
        (s00) edge [bend right] node {} (s01)
        (s01) edge [bend right] node {} (s02)
        (s02) edge [bend right] node {} (s03)
        (s03) edge [bend right] node {} (s04)
        (s04) edge [bend right] node {} (s00)
        ;

        \node[circle, draw=black!60, thick, minimum size=0.5cm, right=3cm of s00] (s0) {\mathtt{1}};
        \node[below=1cm of s0] (pi) {$\displaystyle \pi_2$};
        \node[below left=0.66cm and 1cm of s0] (s1) {\mathtt{2}};
        \node[below right=1cm and 0.13cm of s1] (s2) {\mathtt{3}};
        \node[below right=0.66cm and 1cm of s0] (s4) {\mathtt{6}};
        \node[below left=1cm and 0.13cm of s4] (s3) {\mathtt{7}};
        \path[->]
        (s0) edge [bend right] node {} (s1)
        (s1) edge [bend right] node {} (s2)
        (s2) edge [bend right] node {} (s3)
        (s3) edge [bend right] node {} (s4)
        (s4) edge [bend right] node {} (s0)
        ;
        \end{tikzpicture}

        $\displaystyle P_C(n, k)$ - множество всех циклических $k$-перестановок в $\Sigma$

        $\displaystyle |P_C(n, k)| \cdot k = |P(n, k)|$

        $\displaystyle |P_C(n, k)| = \frac{|P(n, k)|}{k} = \frac{n!}{k(n - k)!}$

        \vspace{5mm}

        \item \textbf{Неупорядоченная расстановка $k$ элементов} (Unordered arrangement of $k$ elements) - мультимножество $\displaystyle \Sigma^*$ размера $k$

        \Ex $\displaystyle \Sigma^* = \Set{\triangle, \triangle, \Box, \triangle, \circ, \Box}^* = \Set{3 \cdot \triangle, 2 \cdot \Box, 1 \cdot \circ} = (\Sigma, r)$

        Неупорядоченную расстановку можно представить как функцию:

        $r : \Sigma \to \Natural, \quad r(x)$ - кол-во повторений объекта $x$

        \vspace{5mm}

        \item \textbf{$k$-сочетание} ($k$-combination) - неупорядоченная перестановка из $k$ различных элементов из $\Sigma$ (еще называют $k$-подмножеством, $k$-subset)

        Соответственно $C(n, k)$ - множество всех таких $k$-сочетаний

        $\displaystyle |C(n, k)| = C^k_n = \begin{pmatrix}n \\ k\end{pmatrix}$

        $C(n, k) = \begin{pmatrix}\Sigma \\ k\end{pmatrix}$

        $\begin{pmatrix}n \\ k\end{pmatrix} \cdot k! = |P(n, k)|$

        $\displaystyle |C(n, k)| = \begin{pmatrix}n \\ k\end{pmatrix} = \frac{n!}{k!(n - k)!}$

    \end{itemize}

    \Th Биномиальная теорема (Binomial theorem):

    \[(x + y)^n = \sum_{k=0}^n \begin{pmatrix}n \\ k\end{pmatrix} x^k y^{n - k}\]

    $\begin{pmatrix}n \\ k\end{pmatrix}$ - биномиальный коэффициент

    \Th Мультиномиальная теорема (Multinomial theorem)

    \[(x_1 + \dots + x_r)^n = \sum_{\substack{k_i \in 1..n, \\ k_1 + \dots + k_r = n}} \begin{pmatrix}n \\ k_1, \dots, k_r\end{pmatrix} x^{k_1}_1 \cdot \dots \cdot x^{k_r}_r\]

    $\displaystyle \begin{pmatrix}n \\ k_1, \dots, k_r\end{pmatrix} = \frac{n!}{k_1! \dots k_r!}$ - мультиномиальный коэффициент



    \Ex мультиномиальной теоремы:

    $\displaystyle (x + y + z)^4 = 1 (x^4 + y^4 + z^4) + 4 (xy^3 + xz^3 + x^3y + yz^3 + y^3z + yz^3) +
    6(x^2y^2 + y^2z^2 + x^2z^2) + 12 (xyz^2 + xy^2z + x^2yz)$

    Доказательство:

    $\Box$

    $\displaystyle (x_1 + \dots + x_r)^n = \sum_{\substack{i_j \in [r] \\ j \in [n]}} x_{i_1}^1 \cdot \dots \cdot x_{i_n}^1 =
    \sum_{\substack{i_j \in [r] \\ j \in [n]}} x_1^{k_1} \cdot \dots \cdot x_r^{k_r}$, где $\displaystyle k_t$ - количество $x$ с индексом $t$ в одночлене ($\displaystyle k_t = |\Set{j \in [n] | i_j = t}|$)

    Получается мультиномиальный коэффицциент $\displaystyle \begin{pmatrix} n \\ k_1, \dots, k_r\end{pmatrix}$
    будет равен количество способов поставить $\displaystyle k_1$ единиц в индексы в $\displaystyle x_{i_1}^1 \cdot \dots \cdot x_{i_n}^1$, $\displaystyle k_2$ двоек в индексы и так далее

    У нас есть $\displaystyle \begin{pmatrix} n \\ k_1\end{pmatrix}$ способов поставить единицу в индексы в одночлен,
    $\displaystyle \begin{pmatrix} n - k_1 \\ k_2\end{pmatrix}$ способов поставить двойку и т. д., получаем:

    $\displaystyle \begin{pmatrix} n \\ k_1, \dots, k_r\end{pmatrix} = \begin{pmatrix} n \\ k_1\end{pmatrix} \begin{pmatrix} n - k_1 \\ k_2\end{pmatrix} \dots \begin{pmatrix} n - k_1 - \dots - k_{r - 1} \\ k_r\end{pmatrix} = [n - k_1 - \dots - k_r = 0] = \\
    \frac{n!}{k_1! (n - k_1)!} \frac{(n - k_1)!}{k_2! (n - k_1 - k_2)!} \dots \frac{(n - k_1 - \dots - k_{r - 1})!}{k_r! 0!} = \frac{n!}{k_1! \dots k_r!}$

    $\Box$

    \noindent \begin{itemize}

        \item \textbf{Перестановка мультимножества $\displaystyle \Sigma^*$} (Permutations of a multiset $\displaystyle \Sigma^*$)

        $\displaystyle \Sigma^* = \Set{\triangle^1, \triangle^2, \Box, \star} = (\Sigma, r) \quad r : \Sigma \to \Natural_0 \quad n = |\Sigma^*| = 4 \quad s = |\Sigma| = 3$

        \Nota \begin{cases}
                  \triangle^1, \triangle^2, \Box, \star \\
                  \triangle^2, \triangle^1, \Box, \star
        \end{cases} считаются равными перестановками

        $\displaystyle |P^*(\Sigma^*, n)| = \frac{n!}{r_1! \dots r_s!} = \begin{pmatrix} n \\ r_1, \dots, r_s\end{pmatrix}$ - количество перестановок мультимножества, где $\displaystyle r_i$ - количество $i$-ого элемента в мультимножестве

        \item \textbf{$k$-комбинация бесконечного мультимножества} ($k$-combinations of infinite multiset) -
        такое субмультимножество размера $k$, содержащее элементы из исходного мультимножества.
        При этом соблюдается, что количество какого-либо элемента $\displaystyle r_i$ в исходном мультимножестве не больше размера комбинации $k$

        $\displaystyle \Sigma^* = \Set{\infty \cdot \triangle, \infty \cdot \Box, \infty \cdot \star, \infty \cdot \Cat}^* \quad n = |\Sigma^*| = \infty$

        $\Sigma = \Set{\triangle, \Box, \star, \Cat} \quad s = |\Sigma| = 4$

        \Ex $5$-комбинация: $\Set{\triangle, \star, \Box, \star, \Box}$

        Разделяем на группы по $\Sigma$ палочками:

        $\triangle \Big| \Box \Box \Big| \star \star \Big| $

        Заменяем элементы на точечки - нам уже не так важен тип элемента, потому что мы знаем из разделения:

        $\bullet \Big| \bullet \bullet \Big| \bullet \bullet \Big| $

        (другой \Exs $\bullet \bullet \bullet \bullet \Big| \Big| \Big| \bullet = \Set{4 \cdot \triangle, 1 \cdot \Cat}$)

        Получается всего $k$ точечек и $s - 1$ палочек, всего $k + s - 1$ объектов. Получаем мультимножество $\Set{k \cdot \bullet, (s - 1) \cdot \Big|}$ (\textit{Star and Bars method})

        Получаем количество перестановок этого мультимножества:
        $\displaystyle \frac{(k + s - 1)!}{k!(s - 1)!} = \begin{pmatrix} k + s - 1 \\ k, s - 1\end{pmatrix} =
        \begin{pmatrix} k + s - 1 \\ k\end{pmatrix} = \begin{pmatrix} k + s - 1 \\ s - 1\end{pmatrix}$

        что и является количеством возможных $k$-комбинаций бесконечного мультимножества

        \vspace{5mm}

        \item \textbf{Слабая композиция} (Weak composition) неотрицательного целого числа $n$ в $k$ частей -
        это решение $\displaystyle (b_1, \dots, b_k)$ уравнение $\displaystyle b_1 + \dots + b_k = n$, где $\displaystyle b_i \geq 0$

        $|\Set{\text{слабая композиция } n \text{ в } k \text{ частей}}| = \begin{pmatrix} n + k - 1 \\ n, k - 1\end{pmatrix}$

        Для решения воспользуемся аналогичным из доказательства мультиномиальной теоремы приемом:

        $n = 1 + 1 + 1 + \dots + 1$

        Поставим палочки:

        $n = 1 + 1 \Big| 1 \Big| \dots + 1$

        Получаем задачу поиска количеств $k$-комбинаций в мультимножестве: $\Set{n \cdot 1, (k - 1) \cdot \Big|}$; получаем $\begin{pmatrix} n + k - 1 \\ n, k - 1\end{pmatrix}$

        \item \textbf{Композиция} (Composition) - решение для $\displaystyle b_1 + \dots + b_k = n$, где $\displaystyle b_i > 0$

        $|\Set{\text{композиция } n \text{ в } k \text{ частей}}| = \begin{pmatrix} n - k + k - 1 \\ n - k, k - 1 \end{pmatrix}$

        Мы знаем, что одну единичку получит каждая $\displaystyle b_i$, поэтому мы решаем это как слабую композицию для $n - k$ в $k$ частей

        \item \textbf{Число композиций $n$ в некоторой число частей} (Number of all compositions into some number of positive parts)

        $\displaystyle \sum_{k=1}^n \begin{pmatrix} n - 1 \\ k - 1 \end{pmatrix} = 2^{n-1}$

        Пусть $t = k - 1$, тогда $\displaystyle \sum_{t = 0}^{n-1} \begin{pmatrix} n - 1 \\ t\end{pmatrix} = 2^{n - 1}$

        \item \textbf{Разбиения множества} (Set partitions) - множество размера $k$ непересекающихся непустых подмножеств

        \begin{tabular}{cp}
            \Exs $\Set{1, 2, 3, 4}, n = 4, k = 2 \rightarrow [\text{разбиение в 2 части}] \rightarrow & \Set{\Set{1}, \Set{2, 3, 4}}, \\
            & \Set{\Set{1, 2}, \Set{3, 4}}, \\
            & \Set{\Set{1, 2, 3}, \Set{4}}, \\
            & \Set{\Set{1, 3}, \Set{2, 4}}, \\
            & \Set{\Set{1, 4}, \Set{2, 3}}, \\
            & \Set{\Set{2}, \Set{1, 3, 4}}, \\
            & \Set{\Set{3}, \Set{1, 2, 4}}$
        \end{tabular}

        $\displaystyle |\Set{\text{разбиение } n \text{ элементов в } k \text{ частей}}| = \begin{Bmatrix} n \\ k \end{Bmatrix} = S^{II}_k (n) = S(n, k)$ - число Стирлинга второго рода

        Для примера выше число Стирлинга $S(4, 2) = \begin{Bmatrix} 4 \\ 2 \end{Bmatrix} = 7$

        Согласно Википедии \href{https://ru.wikipedia.org/wiki/%D0%A7%D0%B8%D1%81%D0%BB%D0%B0_%D0%A1%D1%82%D0%B8%D1%80%D0%BB%D0%B8%D0%BD%D0%B3%D0%B0_%D0%B2%D1%82%D0%BE%D1%80%D0%BE%D0%B3%D0%BE_%D1%80%D0%BE%D0%B4%D0%B0}{для формулы Стирлинга}
        есть формула: $\displaystyle S(n, k) = \frac{1}{k!} \sum_{j=0}^k (-1)^{k+j} \begin{pmatrix}k \\ j\end{pmatrix}j^n$

        \item \textbf{Формула Паскаля} (Pascal's formula)

        $\begin{pmatrix}
             n \\ k
        \end{pmatrix} = \begin{pmatrix}
                            n - 1 \\ k - 1
        \end{pmatrix} + \begin{pmatrix}
                            n - 1 \\ k
        \end{pmatrix}$

        \item \textbf{Рекуррентное отношение для чисел Стирлинга} (Recurrence relation for Stirling$\displaystyle ^{(2)}$ number):

        $\begin{Bmatrix}
             n \\ k
        \end{Bmatrix} = \begin{Bmatrix}
                            n - 1 \\ k - 1
        \end{Bmatrix} + k \cdot \begin{Bmatrix}
                              n - 1 \\ k
        \end{Bmatrix}$

        Возьмем какое-либо разбиение для $n - 1$ элементов на $k$ частей, тогда возможны два случая:

        1) В $k$-ое множество нет ни одного элемента, тогда мы обязаны в него положить наш $n$-ый элемент по определению,
        количество перестановок будет равно $\begin{Bmatrix}n - 1 \\ k - 1\end{Bmatrix} \cdot 1$

        2) В $k$-ом множестве уже есть элементы, тогда все множества будут заполнены и у нас будет выбор из $k$ множеств,
        куда положить $k$-ый элемент, то есть $k \cdot \begin{Bmatrix}n - 1 \\ k\end{Bmatrix}$

        Эти два случая независимы, поэтому получаем $\begin{Bmatrix}n - 1 \\ k - 1\end{Bmatrix} + k \cdot \begin{Bmatrix}n - 1 \\ k\end{Bmatrix}$

        \item \textbf{Число Белла} (Bell number) - количество всех неупорядоченных разбиений множества размера $n$

        Число Белла вычисляется по формуле: $\displaystyle B_n = \sum_{m=0}^n S(n, m)$

        \item \textbf{Целочисленное разбиение} (Integer partition) - решение для $\displaystyle a_1 + \dots + a_k = n$, где $\displaystyle a_1 \geq a_2 \geq \dots \geq a_k \geq 1$

        $p(n, k)$ - число целочисленных разбиений $n$ в $k$ частей

        $\displaystyle p(n) = \sum_{k = 1}^n p(n, k)$ - число всех разбиений для $n$

        \Ex $5 = 5 = 4 + 1 = 3 + 2 = 3 + 1 + 1 = 2 + 2 + 1 = 2 + 1 + 1 + 1 = 1 + 1 + 1 + 1 + 1$


    \end{itemize}




\end{document}

