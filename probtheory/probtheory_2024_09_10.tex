\documentclass[12pt]{article}
\usepackage{preamble}

\pagestyle{fancy}
\fancyhead[LO,LE]{Теория вероятности}
\fancyhead[CO,CE]{10.09.2024}
\fancyhead[RO,RE]{Лекции Блаженова А. В.}

\fancyfoot[L]{\scriptsize исходники найдутся тут: \\ \url{https://github.com/pelmesh619/itmo_conspects} \Cat}

\begin{document}
    \subsection{Построение модели случайных явлений}

    \begin{enumerate}
        \item Задаем пространство элементарных исходов $\Omega$

        \item \Defs Система $\mathcal{F}$ подмножеств $\Omega$ называется $\sigma$-алгеброй событий, если:

        1) $\Omega \in \mathcal{F}$;

        2) $A \in \mathcal{F} \Longrightarrow \overline{A} \in \mathcal{F}$;

        3) $A_1, A_2, \dots, A_n, \dots \in \mathcal{F} \Longrightarrow \bigunion_{i = 1}^\infty A_i \in \mathcal{F}$

        \textbf{Свойства}:

        \begin{enumerate}
            \item $\emptyset \in \mathcal{F}$, так как $\Omega \in \mathcal{F} \Longrightarrow \overline{\Omega} = \emptyset \in \mathcal{F}$

            \item $A_1, A_2, \dots \in \mathcal{F} \Longrightarrow \bigcap_{i = 1}^\infty A_i \in \mathcal{F}$

            \begin{tcolorbox}
                $\Box \quad$ $A_1, A_2, \dots \in \mathcal{F} \Longrightarrow
                \overline{A}_1, \overline{A}_2, \dots \in \mathcal{F} \Longrightarrow
                \bigunion_{i = 1}^\infty \overline{A}_i \in \mathcal{F} \Longrightarrow
                \overline{\bigunion_{i = 1}^\infty \overline{A_i}} = \bigcap_{i = 1}^\infty A_i \in \mathcal{F} \quad \Box$
            \end{tcolorbox}

            \item $A, B \in \mathcal{F} \Longrightarrow A \setminus B \in \mathcal{F}$

            \begin{tcolorbox}
                $\Box \quad$ $A, B \in \mathcal{F} \Longrightarrow A, \overline{B} \in \mathcal{F} \Longrightarrow A \setminus B = A \cdot \overline{B} \in \mathcal{F}$ $\quad \Box$
            \end{tcolorbox}

        \end{enumerate}

        \ExNs{1} $\mathcal{F} = \{\emptyset, \Omega\}$

        \ExNs{2} $\mathcal{F} = \{\emptyset, \Omega, A, \overline{A}\}$

        \ExNs{3} \Defs Борелевская $\sigma$-алгебра $\mathcal{B}(\Real)$ - минимальная $\sigma$-алгебра, содержащая все возможные интервалы на прямой

        % на плоскости борелевская сигма-алгебра - прямоугольники

        \item \Defs $\letsymbol\ \Omega$ - пространство элементарных исходов, $\mathcal{F}$ - его $\sigma$-алгебра событий.
        \textit{Вероятностью} на $(\Omega, \mathcal{F})$ называется функция $P: \mathcal{F} \to \Real$ со свойствами:

        \begin{enumerate}
            \item $P(A) \geq 0 \quad \forall A \in \mathcal{F}$ (неотрицательность)

            \item Если $A_1, A_2, \dots, A_n, \dots \in \mathcal{F}$ - несовместное, то $P(\sum_{i = 1}^\infty A_i) = \sum_{i = 1}^\infty P(A_i)$ (свойство счетной аддитивности)

            \item $P(\Omega) = 1$ (условие нормированности)
        \end{enumerate}

    \end{enumerate}

    \Def Из этого тройка $(\Omega, \mathcal{F}, P)$ называется вероятностным пространством

    \subsection{Свойства вероятности}

    \begin{enumerate}

        \item Так как $\emptyset$ и $\Omega$ - несовместные, то $1 = P(\Omega) = P(\Omega + \emptyset) = 1 + P(\emptyset) \Longrightarrow P(\emptyset) = 0$

        \item Формула обратной вероятности: $P(A) = 1 - P(\overline{A})$

        \begin{tcolorbox}
            $\Box \quad$ $A$ и $\overline{A}$ - несовместные и $A + \overline{A} = \Omega \Longrightarrow P(A + \overline{A}) = P(\Omega) = 1$ $\quad \Box$
        \end{tcolorbox}

        \item $P(A) = 1 - P(\overline{A}) \leq 1$

    \end{enumerate}

    \subsection{Аксиома непрерывности}

    Пусть имеется убывающая цепочка событий $A_1 \supset A_2 \supset A_3 \supset \dots \supset A_n \supset \dots$ и $\bigcap_{i = 1}^\infty A_n = \emptyset$

    Тогда $P(A_n) \underset{n \to \infty}{\to} 0$

    При непрерывном изменении области $A \subset \Omega \subset \Real^n$ соответствующая вероятность $P(A)$ также должна изменятся непрерывно

    \Th Аксиома непрерывности следует из аксиомы счетной аддитивности

    \begin{tcolorbox}
        $\Box$

        Ясно, что $A_n = \sum_{i = n}^\infty A_i \overline{A}_{i + 1} + \prod_{i = n}^\infty A_i$

        $\prod_{i = n}^\infty A_i = A_n \cdot \prod_{i = n + 1}^\infty A_i = \prod_{i = 1}^n
        \cdot \prod_{i = n + 1}^\infty A_i = \prod_{i = 1}^\infty = \emptyset \Longrightarrow
        A_n = \sum_{i = n}^\infty A_n \overline{A_{n + 1}}$ и так как эти события несовместны,
        то по свойству счетной аддитивности $P(A_n) = \sum_{i = n}^\infty P(A_i \overline{A_{i + 1}})$ - это остаток (хвост) сходящегося ряда

        $P(A_1) = \sum_{i = 1}^\infty P(A_i \overline{A_{i + 1}}) = \sum_{i = 1}^{n - 1} P(A_i \overline{A_{i + 1}}) + P(A_n)$ и $P(A_n) \underset{n \to \infty}{\to} 0$ по необходимому признаку сходимости

        $\Box$
    \end{tcolorbox}

    \Nota Аксиому счетной аддитивности можно вывести из конечной аддитивности и аксиомы счетной непрерывности

    \textbf{Свойства операций сложения и умножения}

    1. Свойство дистрибутивности: $A \cdot (B + C) = AB + AC$

    2. Формула сложения: если $A$ и $B$ несовместны, то $P(A + B) = P(A) + P(B)$

    3. Формула сложения вероятностей: $P(A + B) = P(A) + P(B) - P(AB)$

    \begin{tcolorbox}
        $\Box$

        $A + B = A\overline{B} + AB + \overline{A}B$ - несовместные события $\Longrightarrow P(A + B) = P(A\overline{B}) + P(AB) + P(\overline{A}B) =
        (P(A\overline{B}) + P(\overline{A}B)) - P(AB) = P(A) + P(B) - P(AB)$

        $\Box$
    \end{tcolorbox}

    \Ex Из колоды в 36 карт достали одну карту. Какова вероятность того, что будет дама или пика

    Пусть Д - дама, П - пика, $P(\text{Д} + \text{П}) = P(\text{Д}) + P(\text{П}) - P(\text{Д}\text{П}) = \frac{4}{36} + \frac{9}{36} - \frac{1}{36} = \frac{1}{3}$

    Формула сложения при $N = 3$: $P(A_1 + A_2 + A_3) = P(A_1) + P(A_2) + P(A_3) - P(A_2 A_3) - P(A_1 A_3) - P(A_1 A_2) + P(A_1 A_2 A_3)$

    Общий случай: $P(A_1 + A_2 + \dots + A_n) =  \sum_{i = 1}^n P(A_i) - \sum_{i < j} P(A_i A_j) + \sum_{i < j < k} P(A_i A_j A_k) + (-1)^{n - 1} \cdot P(A_1 A_2 \dots A_n)$ - формула включения и исключения

    \Ex $n$ писем случайно раскладывается по $n$ конвертам. Найти вероятность того, что хотя бы одно письмо окажется в своем конверте

    $\letsymbol A_i$ - $i$-ое письмо в своем конверте

    $P(A_i) = \frac{1}{n}; P(A_i A_j) = \frac{1}{A^2_n}; P(A_i A_j A_k) = \frac{1}{A^3_n}; P(A_1 A_2 \dots A_n) = \frac{1}{n!}$

    Слагаемых вида $A_i$ - $n$ штук; $A_i A_j$ - $C^2_n$; $A_i A_j A_k$ - $C^3_n$; $A_1 A_2 \dots A_n$ - 1 штука

    $P(A) = P(A_1 + A_2 + \dots + A_n) = n \cdot \frac{1}{n} - C^2_n \frac{1}{A^2_n} + C^3_n \frac{1}{A^3_n} - \dots + (-1)^{n - 1} \frac{1}{n!} = 1 - \frac{1}{2} + \frac{1}{3!} - \dots + (-1)^{n - 1} \frac{1}{n!}$

    Так как $e^{-1} = 1 - 1 + \frac{1}{2} - \frac{1}{3!} + \dots$, то при $n \to \infty \quad P(A) \underset{n \to \infty}{\to} 1 - e^{-1} \approx 0.63$

    Независимые события

    Под независимыми событиями логично подразумевать события, не связанные причинно-следственной связью (то есть когда факт наступления одного не влияет на оценку вероятности другого)

    $\letsymbol |\Omega| = n; |A| = m_1; |B| = m_2$

    Проведем пару независимых испытаний. Тогда получаем пространство элементарных исходов $\Omega \times \Omega$ и $|\Omega \times \Omega| = n^2$

    По основному принципу комбинаторики $|A \cdot B| = m_1 \cdot m_2$

    $P(AB) = \frac{|A \cdot B|}{|\Omega \times \Omega|} = \frac{m_1 m_2}{n^2} = P(A) \cdot P(B)$

    \Def События $A$ и $B$ называются независимыми, если $P(A \cdot B) = P(A) \cdot P(B)$

    \Lab $\letsymbol P(A), P(B) \neq 0$, доказать, что если $A$ и $B$ несовместны, то они зависимы

    Свойство: Если $A$ и $B$ независимы, то независимы $\overline{A}$ и $\overline{B}$, $A$ и $\overline{B}$, $\overline{A}$ и $B$

    Доказательство: $A = A \cdot (B + \overline{B}) = AB + A\overline{B}$ - несовместные события $\Longrightarrow P(A) = P(AB) + P(A\overline{B}) \Longrightarrow P(A\overline{B}) = P(A) - P(AB) =
    P(A) - P(A) \cdot P(B) = P(A) (1 - P(B)) = P(A) P(\overline{B}) \Longrightarrow$ независимы

    \Def События $A_1, A_2, \dots A_n$ - независимы в совокупности, если для любого набора $i_1, i_2, \dots, i_k \ (2 \leq k \leq n)$
    $P(A_{i_1} \cdot A_{i_2} \cdot \dots \cdot A_{i_k}) = P(A_{i_1}) \cdot P(A_{i_2}) \cdot \dots \cdot P(A_{i_k})$

    \Nota Из независимости в совокупности при $k = 2$ получаем попарную независимость. Обратное утверждение неверно

    \Ex (С. Бернштейн)

    Пусть имеется правильный тетраэдр, одна грань окрашена в красный, вторая в синий, третья в зеленый, а четвертая во все эти три цвета.

    Подбросили тетраэдр, $\letsymbol A$ - грань, которая содержит красный цвет, $B$ - синий, $C$ - зеленый.

    $P(A) = P(B) = P(C) = \frac{2}{4} = \frac{1}{2}$

    Так как $P(AB) = P(AC) = P(BC) = \frac{1}{4}$

    $P(AB) = \frac{1}{4} = \frac{1}{2} \cdot \frac{1}{2} = P(A) P(B)$ - попарная независимость

    $P(ABC) = \frac{1}{4} \neq P(A) P(B) P(C)$ - но вот независимость в совокупности не соблюдается

    \Ex (Шевалье де Мере, Паскаль, Ферма, $\approx$ 1650 г.)

    Какова вероятность того, что при 4 бросании кости выпадет одна шестерка

    $A_1$ - при первом броске шестерка, $A_2$ - при втором, $A_3$ - при третьем, $A_4$ - при четвертом

    $B$ - выпала хотя бы одна шестерка при 4 бросках

    $B = A_1 + A_2 + A_3 + A_4$ - совместные события, но независимые

    Найдем обратную вероятность: $\overline{B}$ - ни разу не выпала шестерка

    $\overline{B} = \overline{A_1} \cdot \overline{A_2} \cdot \overline{A_3} \cdot \overline{A_4}$

    $P(\overline{A_1}) = P(\overline{A_2}) = P(\overline{A_3}) = P(\overline{A_4}) = \frac{5}{6}$

    $\overline{B} = P(\overline{A_1}) P(\overline{A_2}) P(\overline{A_3}) P(\overline{A_4}) = \left(\frac{5}{6}\right)^4 \approx 0.482$

    $P(B) = 1 - P(\overline{B}) \approx 0.52$

\end{document}
