\documentclass[12pt]{article}
\usepackage{preamble}

\pagestyle{fancy}
\fancyhead[LO,LE]{Дополнительные главы \\ высшей математики}
\fancyhead[CO,CE]{04.10.2024}
\fancyhead[RO,RE]{Лекции Далевской О. П.}

\fancyfoot[L]{\scriptsize исходники найдутся тут: \\ \url{https://github.com/pelmesh619/itmo_conspects} \Cat}

\begin{document}
    \begin{MyProof}
        $\Box$

        $\sum_{n = 1}^\infty (-1)^{n + 1} u_n = u_1 - u_2 + u_3 - u_4 + \dots + u_n + \dots$

        $S_{2n} = (u_1 - u_2) + (u_3 - u_4) + \dots + (u_{2n - 1} - u_{2n})$

        Все слагаемые в скобках будут больше нуля, тогда частичные суммы будут возрастать

        $S_{2n} = u_1 - (u_2 - u_3) - (u_4 - u_5) - \dots - (u_{2n - 2} - u_{2n - 1}) - u_{2n} < u_1$

        Здесь же тоже все слагаемые больше нуля - их мы вычитаем из $u_1$ и получаем число гарантированно меньшее $u_1$

        По \Ths о монотонности и ограниченности последовательность $\letsymbol \lim_{n \to \infty} S_{2n} = S \in \Real$

        $S_{2n + 1} = S_{2n} + u_{2n + 1}; \qquad \lim_{n \to \infty} S_{2n + 1} = \lim_{n \to \infty} S_{2n} + \cancelto{0}{\lim_{n \to \infty} u_{2n + 1}} = S \in \Real$

        $\Box$
    \end{MyProof}

    \Exs $\sum_{n = 1}^\infty \frac{(-1)^{n + 1}}{n} = 1 - \frac{1}{2} + \frac{1}{3} - \frac{1}{4} + \dots$

    $u_n = \frac{1}{n} \underset{n \to \infty}{\longrightarrow} 0, \qquad \frac{1}{n} > \frac{1}{n + 1} \Longrightarrow$ ряд сходится

    \hypertarget{seriesremainderevaluation}{}

    \Nota Оценка остатка ряда

    Запишем ряд: $\sum_{n = 1}^\infty (-1)^{n + 1} u_n = u_1 - u_2 + u_3 - \dots \pm u_n \mp u_{n + 1} =
    S + \sum_{k = n + 1}^\infty (-1)^{k} u_k = S_n + \underset{\uparrow \text{ остаток ряда}}{R_n\phantom{ikkkkkkkk}}$

    В доказательстве \Ths было установлено, что сумма ряда не превышает своего первого члена

    \[R_{n + 1} < |(-1)^{k + 1} u_k| = u_k = u_{n + 1}\]

    \Ex $1 - \frac{1}{2} + \frac{1}{4} - \frac{1}{8} + \frac{1}{16} \underset{R_4}{\underbrace{ - \frac{1}{32} + \dots}} = \sum_{n = 0}^\infty (-1)^n \frac{1}{2^n}$

    $|R_4| < \frac{1}{32}$

    Проверка: $-\left(\frac{1}{32} - \frac{1}{64}\right) - \left(\frac{1}{128} - \frac{1}{256}\right) - \dots = -\sum_{k = 3}^\infty \frac{1}{2^{2k}}$ - \Lab досчитать и сравнить с $\frac{1}{32}$

    \Nota Оценка не работает в знакоположительных рядах

    $1 + \frac{1}{2} + \frac{1}{4} + \frac{1}{8} + \frac{1}{16} + \dots$

    $R_3 = \frac{1}{16} + \frac{1}{32} + \frac{1}{64} + \dots = \frac{1}{16} \left(1 + \frac{1}{2} + \dots\right) = \frac{2}{16} = \frac{1}{8} > \frac{1}{16}$

    \hypertarget{changingsignseries}{}

    \Def Знакопеременный ряд

    $\sum_{n = 1}^\infty u_n$, где $u_n$ - любого знака и не все $u_n$ одного знака

    \Ex $1 - \frac{1}{2} + \frac{1}{3} + \frac{1}{4} - \frac{1}{5} - \frac{1}{6} + \frac{1}{7} + \dots$

    \Nota Исследование таких рядов (в том числе знакочередующихся) на сходимость можно проводить при помощи ряда из модулей

    \hypertarget{absoluteconvergence}{}

    \begin{MyTheorem}
        \Ths Абсолютная сходимость

        $\sum_{n = 1}^\infty |u_n|$ сходится $\Longrightarrow \sum_{n = 1}^\infty u_n$ сходится
    \end{MyTheorem}

    \Mems См. абсолютную сходимость в \href{https://pelmesh619.github.io/itmo_conspects/conspects/calculus/calculus_superconspect.pdf}{несобственных интегралах}

    \begin{MyProof}
        $\Box$

        По критерию Коши:

        $\sum_{n = 1}^\infty |u_n|$ сходится $\Longleftrightarrow$

        $\forall \varepsilon > 0 \ \exists \underset{n_0 = n_0(\varepsilon)}{n_0 \in \Natural} \ | \ \forall m > n > n_0 \quad ||u_n| + |u_{n + 1}| + \dots + |u_m|| < \varepsilon$

        По неравенству треугольника:

        $|u_n| + |u_{n + 1} + \dots + u_m| < |u_n| + |u_{n + 1}| + \dots + |u_m| < \varepsilon$

        $\Box$
    \end{MyProof}

    \Notas Обратное неверно!

    \Ex $\sum_{n = 1}^\infty \frac{(-1)^{n + 1}}{n} = 1 - \frac{1}{2} + \frac{1}{3} + \dots $ сходится

    Но $\sum_{n = 1}^\infty \frac{1}{n}$ расходится

    \Def Если $\sum u_n$ сходится, при том что $\sum |u_n|$ сходится, он называется \textbf{абсолютно сходящимся}

    \hypertarget{conditionalconvergence}{}

    \Defs Если $\sum u_n$ сходится, при том что $\sum |u_n|$ расходится, он называется \textbf{условно сходящимся}

    \Notas Для абсолютно сходящихся рядов перестановка членов безболезнена и сохраняет сумму ряда

    \Notas На абсолютно сходящиеся ряды распространяются признаки сходимости знакоположительных рядов

    1) Признак сравнения: $|u_n| < |v_n|: \ \sum |v_n|$ сходится $\Longrightarrow \sum |u_n|$ сходится

    2) Предельный признак: $\lim \left|\frac{u_n}{v_n}\right| = q \in \Real \setminus \{0\}$

    3) $D = \lim \left|\frac{u_{n + 1}}{u_n}\right| < 1$

    4) $K = \lim \sqrt[n]{|{u_n}|} < 1$

    5) $\int_a^\infty f(x)dx$ сравнивается с $\sum |u_n|$

    \clearpage


    \section{\S 2. Функциональные ряды}

    \hypertarget{functionalseries}{}

    \subsection{1. Определения}

    \Def $\sum_{n = 1}^\infty u_n(x)$, где $u_n(x) : E \subset \Real \to \Real$ называется функциональным

    \Notas При фиксации переменной $x$ ряд становится числовым

    \Exs $\sum_{n = 0}^\infty x^n$

    $x = 2 \quad \sum_{n = 0}^\infty 2^n$ расходится

    $x = \frac{1}{2} \quad \sum_{n = 0}^\infty \left(\frac{1}{2}\right)^n$ сходится

    Таким образом для $|x| < 1$ ряд будет сходящимся, для $|x| > 1$ расходящимся

    \Def Множество значений $x$, при которых $\sum_{n = 1}^\infty u_n(x)$ сходится, называется областью сходимости

    \hypertarget{functionalseriesconvergence}{}

    \Defs Если ряд $\sum_{n = 1}^\infty u_n(x)$ сходится при всех $x$ из некоторого множества $E$, то сумма ряда -
    функция $S(x)$

    \Nota То есть $\exists \lim_{n \to \infty} S_n(x) = S(x)$

    Запишем остаток: $R_n(x) = S(x) - S_n(x)$. Часто удобно исследовать $R_n(x) \to 0$. Также работает критерий Коши

    \begin{MyTheorem}
        \Ths Критерий Коши

        $\sum_{n = 1}^\infty u_n(x)$ сходится в области $D \Longleftrightarrow
        \forall \varepsilon > 0 \ \exists \underset{n_0 = n_0(\varepsilon, x)}{n_0 \in \Natural} \ | \
        \forall m > n > n_0 \ |u_n(x) + u_{n + 1}(x) + \dots + u_m(x)| < \varepsilon$
    \end{MyTheorem}

    \Notas Очень неприятно, что $n_0$ зависит от $\varepsilon$ и всякого $x$

    \hypertarget{uniformconvergence}{}

    \Def Равномерная сходимость ряда

    $\sum_{n = 1}^\infty u_n(x)$ равномерно сходится в области $D \overset{def}{\Longleftrightarrow}$

    $\forall \varepsilon > 0 \ \exists \underset{n_0 = n_0(\varepsilon)}{n_0 \in \Natural} \ | \ \forall n > n_0 \ |R_n(x)| < \varepsilon$

    \Nota Доказательства равномерной сходимости по определению сложно, пользуются другими способами

    \hypertarget{weierstrassign}{}

    \begin{MyTheorem}
        \Ths Признак Вейерштрасса

        $\exists \sum_{n = 1}^\infty \alpha_n$ - числовой ряд такой, что $\alpha_n > 0$, $\sum \alpha_n$ сходится,
        $|u_n(x)| \leq \alpha_n \ \forall n$

        Тогда $\sum_{n = 1}^\infty u_n(x)$ равномерно сходящийся

    \end{MyTheorem}

    \hypertarget{majorseries}{}

    \Nota Ряд $\sum_{n = 1}^\infty \alpha_n$ называется мажорирующим (то есть преобладающий), 
    а ряд $\sum_{n = 1}^\infty u_n(x)$ - мажорируемым

    \begin{MyProof}
        $\Box$

        $\sum_{n = 1}^\infty \alpha_n$ сходится $\Longleftrightarrow \forall \varepsilon > 0 \ \exists \underset{n_0 = n_0(\varepsilon)}{n_0 \in \Natural} \ | \
        \forall n > n_0 \ |R_n^\alpha| < \varepsilon$

        Заменим на условие $|\alpha_n + \dots + \alpha_m| < \varepsilon$ (кр. Коши)

        $|u_n(x) + \dots + u_m(x)| \leq |u_n(x)| + \dots + |u_m(x)| \leq \alpha_n + \dots + \alpha_m \leq \varepsilon$

        При этом номер $n_0$ зависит только от $\varepsilon$

        $\Box$
    \end{MyProof}

    \Nota Таким образом всякий мажорируемый ряд равномерно сходится, но не всякий равномерно сходящийся ряд мажорируем

    \Nota Установим свойство суммы равномерно сходящегося ряда

    \Exs Рассмотрим ряд $\sum_{n = 1}^\infty (x^{\frac{1}{2n + 1}} - x^{\frac{1}{2n - 1}}) = (x^\frac{1}{3} - x^1) + (x^\frac{1}{5} - x^\frac{1}{3}) + (x^\frac{1}{7} - x^\frac{1}{5}) + \dots;$

    $S_n = x^\frac{1}{2n + 1} - x$

    При $x > 0 \quad \lim_{n \to \infty} S_n = \lim_{n \to \infty} (x^\frac{1}{2n + 1} - x) = 1 - x$

    При $x < 0 \quad \lim_{n \to \infty} S_n = \lim_{n \to \infty} (-\sqrt[2n + 1]{|x|} - x) = -1 - x$

    При $x = 0 \quad S_n = 0$



\end{document}

