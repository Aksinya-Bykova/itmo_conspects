\clearpage

\section{X. Программа экзамена в 2024/2025}


\begin{enumerate}
\subsection{X.1. Числовые ряды.}

    \item Определение числового ряда, понятие суммы ряда.

    \hyperlink{numberseriesdefinition}{Определение числового ряда}: $\{u_1, u_2, \dots, u_n, \dots\} = \{u_n\}$ называется числовым рядом

    $u_n$ называется общим членом ряда
    
    \hyperlink{sumofseriesdefinition}{Понятие суммы ряда}: Частичная сумма ряда $S_n \stackrel{def}{=} \sum_{k = 1}^{n} u_k$

    Если $\exists \lim_{n \to\infty} S_n = S \in \Real$, то ряд $\sum_{n = 1}^\infty u_n$ называют сходящимся,
    а $S$ называют суммой ряда $\sum_{n = 1}^\infty u_n = S$

    \item Сходимость числового ряда. Эталонные ряды: геометрический, гармонический.

    \hyperlink{seriesconvergence}{Сходимость числового ряда}: Если $\exists \lim_{n \to\infty} S_n = S \in \Real$, то ряд $\sum_{n = 1}^\infty u_n$ называют сходящимся

    \hyperlink{geometricseries}{Геометрический ряд}: $\sum_{n = 0}^\infty b q^n$ - сходится при $|q| < 1$, тогда $S = \frac{b}{1 - q}$
    
    \hyperlink{harmonicseries}{Гармонический ряд}: $\sum_{n = 1}^\infty \frac{1}{n}$ - расходится

    \item Условия сходимости рядов: необходимое условие, критерий Коши.

    \hyperlink{necessarycondition}{Необходимое условие}: 

    \begin{MyTheorem}
        \Ths Если $\sum_{n = 1}^\infty u_n$ сходится, то верно, что $\lim_{n \to \infty} u_n = 0$
    \end{MyTheorem}

    \hyperlink{cauchycriteria}{Критерий Коши}:

    \begin{MyTheorem}
        $\sum_{n = 1}^\infty u_n$ сходится $\Longleftrightarrow \forall \varepsilon > 0 \ \exists \underset{n_0 = n_0 (\varepsilon)}{n_0 \in \Natural} \ | \ \forall m > n > n_0 \ \ \underset{|S_m - S_n| < \varepsilon}{|u_{n + 1} + \dots + u_m|} < \varepsilon$
    \end{MyTheorem}


    \item Знакоположительные числовые ряды, свойства.
    \item Признаки сходимости знакоположительных числовых рядов: признаки сравнения.
    \item Признак Даламбера, радикальный признак Коши.
    \item Интегральный признак сходимости.
    \item Знакочередующиеся ряды. Теорема Лейбница. Оценка остатка ряда.
    \item Знакопеременные ряды. Абсолютная и условная сходимость.

\subsection{X.2. Функциональные ряды.}

    \item Функциональные ряды. Сходимость. Поточечная и равномерная сходимость ряда.
    Мажорирующий ряд.
    \item Признак Вейерштрасса.
    \item Непрерывность суммы ряда.
    \item Свойства равномерно сходящихся рядов (дифференцирование и интегрирование суммы
    ряда).
    \item Степенные ряды. Теорема Абеля. Радиус сходимости.
    \item Ряд Тейлора. Стандартные разложения элементарных функций.
    \item Ортогональные системы функций и ряды Фурье. Определение тригонометрического ряда
    Фурье для функции на отрезке $[-\pi, \pi]$. Теорема Дирихле.
    \item Тригонометрический ряд Фурье на произвольном отрезке (сдвиг, растяжение)
\end{enumerate}