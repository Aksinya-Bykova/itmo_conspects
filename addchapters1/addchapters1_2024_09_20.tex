\documentclass[12pt]{article}
\usepackage{preamble}

\pagestyle{fancy}
\fancyhead[LO,LE]{Дополнительные главы \\ высшей математики}
\fancyhead[CO,CE]{20.09.2024}
\fancyhead[RO,RE]{Лекции Далевской О. П.}

\fancyfoot[L]{\scriptsize исходники найдутся тут: \\ \url{https://github.com/pelmesh619/itmo_conspects} \Cat}

\begin{document}
    \subsubsection{3.3. Достаточное условие (признаки сходимости)}

    Здесь мы рассмотрим:

    \begin{enumerate}
        \item Признак сравнения (в неравенствах)
        \item Предельный признак сравнения
        \item Признак Даламбера
        \item Признак Коши (радикальный)
        \item Признак Коши (интегральный)
    \end{enumerate}

    Далее $\sum_{n = 1}^\infty u_n$ - исследуемый ряд, $\sum_{n = 1}^\infty v_n$ - вспомогательный (уже исследован на сходимость),
    для простоты $v_n, u_n > 0$ (для отрицательных доказывается аналогично)

    \mediumvspace

    \hypertarget{comparisonsign}{}

    \begin{MyTheorem}
        \ThNs{1} Признак сравнения (в неравенствах)

        a) $\letsymbol 0 < u_n \leq v_n. \quad$ Тогда $\sum v_n$ сходится $\Longrightarrow \sum u_n$ сходится

        б) $\letsymbol 0 \leq v_n \leq u_n. \quad$ Тогда $\sum v_n$ расходится $\Longrightarrow \sum u_n$ расходится
    \end{MyTheorem}

    \begin{MyProof}
        $\Box$

        а) Строим частичные суммы:

        $\sum v_n$ сходится $\Longleftrightarrow \exists \lim_{n \to \infty} \sigma_n = \sigma \in \Real$

        $S_n, \sigma_n$ возрастают и обе ограничены числом $\sigma$

        Следовательно $\exists \lim_{n \to \infty} S_n = S \leq \sigma$

        Аналогично пункт б)

        $\Box$
    \end{MyProof}

    \hypertarget{limitcomparisonsign}{}

    \begin{MyTheorem}
        \ThNs{2} Предельный признак

        $\lim_{n \to \infty} \frac{u_n}{v_n} = q \in \Real \setminus \{0\} \ \Longrightarrow \
        \begin{sqcases}
            \sum u_n \text{ сходится, если } \sum v_n \text{ сходится} \\
            \sum u_n \text{ расходится, если } \sum v_n \text{ расходится}
        \end{sqcases}$
    \end{MyTheorem}

    \begin{MyProof}
        $\Box$

        По определению предела

        $\exists \lim_{n \to \infty} \frac{u_n}{v_n} = q \in \Real \Longleftrightarrow \forall \varepsilon > 0 \ \exists n_0 \in \Natural \ | \ \forall n > n_0 \ \left|\frac{u_n}{v_n} - q\right| < \varepsilon$

        $\left|\frac{u_n}{v_n} - q\right| < \varepsilon \Longleftrightarrow q - \varepsilon < \frac{u_n}{v_n} < q + \varepsilon$

        $(q - \varepsilon) v_n < u_n < (q + \varepsilon) v_n$

        а) Если $\sum v_n$ сходится, то из правой части неравенства: $0 < u_n < (q + \varepsilon) v_n$

        По признаку сравнения $\sum u_n$ также сходится

        б) Если $\sum v_n$ расходится, то из левой части неравенства: $0 < (q - \varepsilon) v_n < u_n$

        Тогда по пункту б) \ThNs{1} $\sum u_n$ расходится

        $\Box$
    \end{MyProof}

    \Nota При $q = 0$ можем говорить, что $u_n$ - бесконечно малая высшего порядка, чем $v_n$, а значит, если ряд $v_n$
    сходится, то $u_n$ сходится

    \ExN{1} $\sum_{n = 1}^\infty \frac{1}{n^2} = \sum_{n = 1}^\infty u_n$

    $\sum_{n = 1}^\infty \frac{1}{n(n + 1)} = \sum_{n = 1}^\infty v_n$ сходится

    $\frac{1}{n(n + 1)} = \frac{1}{n^2 + n} > \frac{1}{n^2 + 2n + 1} = \frac{1}{(n + 1)^2}$

    $\sum_{n = 0}^\infty \frac{1}{(n + 1)^2} = \sum_{n = 1}^\infty \frac{1}{n^2}$ сходится по признаку сравнения

    \ExN{2} $\sum_{n = 1}^\infty \frac{1}{n!} = \sum_{n = 1}^\infty u_n$

    $\sum_{n = 1}^\infty \frac{1}{2^n} = \sum_{n = 1}^\infty v_n$ сходится

    Начиная с некоторого $n_0$ $n! > 2^n$. Тогда $u_n < v_n$ при $n > n_0$, по признаку сравнения $\sum_{n = 1}^\infty \frac{1}{n!}$ сходится


    \ExN{3} $\sum_{n = 1}^\infty \arcsin \frac{1}{n}$

    \Nota Члены рядов $u_n$ и $v_n$ - бесконечно малые последовательности. Иначе ряды расходятся по необходимому условию.
    Тогда в \ThNs{2} сравниваются порядки бесконечно малых, и ряды одновременно сходятся или расходятся, если $u_n$ и $v_n$ одного
    порядка малости. По этому принципу подбирается вспомогательный ряд

    $u_n = \arcsin \frac{1}{n} \underset{n \to \infty}{\sim} \frac{1}{n} = v_n \quad\quad \sum_{n = 1}^\infty v_n = \sum_{n = 1}^\infty \frac{1}{n}$ расходится

    \hypertarget{dalambersign}{}

    \begin{MyTheorem}
        \ThNs{3} Признак Даламбера

        $\sum_{n = 1}^\infty u_n$ - исследуемый, $\exists \mathcal{D} = \lim_{n \to \infty} \frac{u_{n + 1}}{u_n} \in \Real^+$

        \begin{tabular}{ll}
            а) $0 \leq \mathcal{D} < 1$ & $\Longrightarrow \sum u_n$ сходится                                \\

            б) $\mathcal{D} > 1$        & $\Longrightarrow \sum u_n$ расходится                              \\

            в) $\mathcal{D} = 1$        & $\Longrightarrow$ ничего не следует, требуется другое исследование \\
        \end{tabular}
    \end{MyTheorem}

    \begin{MyProof}
        $\Box$

        а) По определению предела $\mathcal{D} = \lim_{n \to \infty} \frac{u_{n + 1}}{u_n}, \ 0 \leq \mathcal{D} < 1 \Longleftrightarrow$

        $\forall \varepsilon > 0 \ \exists n_0 \in \Natural \ | \ \forall n > n_0 \ \left|\frac{u_{n + 1}}{u_n} - \mathcal{D}\right| < \varepsilon \quad \Longleftrightarrow \quad \mathcal{D} - \varepsilon < \frac{u_{n + 1}}{u_n} < \mathcal{D} + \varepsilon$

        Так как $0 \leq \mathcal{D} < 1$, можно втиснуть число $r$ между $\mathcal{D}$ и 1: $\mathcal{D} < r < 1$

        Положим $\varepsilon = r - \mathcal{D}$, то есть $\mathcal{D} + \varepsilon = r$

        Смотрим правую часть $\frac{u_{n + 1}}{u_n} < r$ для $\forall n > n_0$, где $n_0 = n_0(\varepsilon), \varepsilon = r - \mathcal{D}$

        $u_{n_0 + 1} < r u_{n_0}$

        $u_{n_0 + 2} < r u_{n_0 + 1} < r^2 u_{n_0}$

        $u_{n_0 + l} < r^l u_{n_0}$

        $\sum_{n = 1}^\infty u_n = \underset{k}{\underbrace{u_1 + u_2 + \dots + u_{n_0 - 1}}} + u_{n_0} + \dots = k + \sum_{m = 1}^\infty v_m $ % u_{n_0} + u_{n_0 + 1} + \dots

        Члены $v_m < r^l u_{n_0}$; $u_{n_0}$ - фикс. число, а $\sum_{l = 1}^\infty r^l$ сходится как геометрический при $|r| < 1$

        Итак ряд $\sum_{l = 1}^\infty r^l u_{n_0}$ сходится и почленно превышает $\sum v_m = \left(\sum u_n\right) - k$

        То есть $\sum u_n$ сходится

        б) \Lab (взять $r$ между $\mathcal{D}$ и $1$, $1 < r < \mathcal{D}$, $\mathcal{D} - r = \varepsilon$)

        Сравнить $\sum u_n$ с $\sum r^l$ (расходящимся)

        $\Box$
    \end{MyProof}

    \ExN{1} $\sum_{n = 1}^\infty \frac{1}{n!} \quad\quad \mathcal{D} = \lim_{n \to \infty} \frac{u_{n + 1}}{u_n} = \lim_{n \to \infty} \frac{n!}{(n + 1)!} = \lim_{n \to \infty} \frac{1}{n + 1} = 0$ - сходится

    \ExN{2} $\sum_{n = 1}^\infty \frac{1}{n} \quad\quad \mathcal{D} = \lim_{n \to \infty} \frac{n}{n + 1} = 1$ - расходится

    $\sum_{n = 1}^\infty \frac{1}{n^2} \quad\quad \mathcal{D} = \lim_{n \to \infty} \frac{n^2}{(n + 1)^2} = 1$ - сходится

    \hypertarget{cauchyradicalsign}{}

    \begin{MyTheorem}
        \ThNs{4} Радикальный признак Коши

        $\sum_{n = 1}^\infty u_n \quad\quad u_n \geq 0$ и $\exists \lim_{n \to \infty} \sqrt[n]{u_n} = K \in \Real$

        а) $0 \leq K < 1 \Longrightarrow \sum u_n$ сходится

        б) $K > 1 \Longrightarrow \sum u_n$ расходится
    \end{MyTheorem}

    \Notas $K = 1$ - ничего не следует

    \begin{MyProof}
        $\Box$

        а) По определению предела $\forall \varepsilon > 0 \ \exists n_0 \in \Natural \ | \ \forall n > n_0 \ |\sqrt[n]{u_n} - K| < \varepsilon$

        $\Longleftrightarrow K - \varepsilon < \sqrt[n]{u_n} < K + \varepsilon$ Положим $\varepsilon = r - K$, где $K < r < 1$

        $\Longrightarrow 0 \leq u_n < r^n$ - геом. ряд с $|r| < 1$, то есть $\sum r^n$ сходится

        б) Аналогично

        $\Box$
    \end{MyProof}

    \ExN{1} $\sum_{n \to \infty} \left(\frac{n}{n + 1}\right)^n \quad\quad K = \lim_{n \to \infty} \left(\frac{n}{n + 1}\right)^n = \lim_{n \to \infty} \frac{n}{n + 1} = 1$

    $\mathcal{D} = \lim_{n \to \infty} \frac{\left(1 - \frac{1}{n + 2}\right)^{n + 1}}{\left(1 - \frac{1}{n + 1}\right)^{n}}$

    Но $\lim_{n \to \infty} u_n = e^{-1} \neq 0$ - необходимое условие не выполняется

    \ExN{2} $\sum_{n = 1}^\infty \left(\frac{n}{n + 1}\right)^{n^2}, \quad\quad K = \lim_{n \to \infty} \sqrt[n]{\left(\frac{n}{n + 1}\right)^n} = e^{-1} < 1$ - сходится

    \hypertarget{cauchyintegralsign}{}

    \begin{MyTheorem}
        \ThNs{5} Интегральный признак Коши

        Если существует $f(x) : [1; +\infty] \to \Real^+, f(x)$ монотонно убывает, $f(n) = u_n$, то $\sum_{n = 1}^\infty u_n$ и $\int_{1}^\infty f(x) dx$ одновременно сходятся или расходятся
    \end{MyTheorem}

    \begin{MyProof}
        $\Box$

        $\int_1^{+\infty} f(x) dx = \lim_{b \to \infty} \int_1^b f(x) dx$

        $\sum_{n = 2}^b u_n = u_2 \cdot 1 + u_3 \cdot 1 + \dots < \int_1^b f(x) dx < u_1 \cdot 1 + u_2 \cdot 1 + \dots = \sum_{n = 1}^{b - 1} u_n$

        Обозначим $\sum_{n = 1}^{b - 1} u_n = S_{b - 1}, \quad \sum_{n = 2}^{b} u_n = S_{b - 1} - u_1 + u_b$

        $0 < S_{b - 1} - u_1 + u_b < \int_1^b f(x) dx < S_{b - 1}$

        $0 < \sum_{n = 1}^{\infty} u_n - u_1 + u_b < \int_1^\infty f(x) dx < \sum_{n = 1}^\infty u_n$

        Если $\int$ сходится, то смотрим левую часть

        Если $\int$ расходится, то смотрим правую часть неравенства

        $\Box$
    \end{MyProof}

    \hypertarget{alternatingsignseries}{}

    \subsection{4. Знакочередующиеся ряды}

    \Def $\sum_{n = 0}^\infty (-1)^n u_n$ ($u_n > 0$) - знакочередующийся ряд

    \hypertarget{leibniztheorem}{}

    \begin{MyTheorem}
        \Ths Признак Лейбница

        Если для знакочередующегося ряда $\sum_{n = 0}^\infty (-1)^n u_n$ верно,
        что \ $\underset{n \to \infty}{u_n \to 0}$ \ и \ $|u_1| > |u_2| > \dots > |u_n|$,
        то ряд $\sum_{n = 0}^\infty (-1)^n u_n$ сходится
    \end{MyTheorem}

\end{document}

