\documentclass[12pt]{article}
\usepackage{preamble}

\pagestyle{fancy}
\fancyhead[LO,LE]{$\mathcal{D}$искретная математика}
\fancyhead[CO,CE]{14.05.2024}
\fancyhead[RO,RE]{Лекции Чухарева К. И.}



\begin{document}
    \begin{itemize}
        \item \textbf{Принцип включения/исключения} (Principle of Incusion/Exclusion (PIE))

        $|A \union B \union C| = |A| + |B| + |C| - |A \cap B| - |B \cap C| - |A \cap C| + |A \cap B \cap C|$

        \Ex есть $n = 11$ объектов, нужно распределить их между $k = 3$ группами $A$, $B$ и $C$

        Эту задачу можно решить с помощью \textit{Stars and bars method}, тогда мы получим $
        \begin{pmatrix} n + k - 1 \\ n, k - 1 \end{pmatrix} = \begin{pmatrix} 13 \\ 2 \end{pmatrix} = 78$

        Введем ограничение: пусть мощность каждого множества будет не больше 4.

        Посчитаем количество неподходящих вариантов:

        $|A| = |\Set{b_A \geq 5}| = 1 \cdot
        \begin{pmatrix} 11 - 5 + 3 - 1 \\ 3 - 1 \end{pmatrix} =
        \begin{pmatrix} 8 \\ 2 \end{pmatrix} = 28$

        $|A \cap B| = |\Set{b_A \geq 5 \land b_B \geq 5}| =
        \begin{pmatrix} 3 \\ 2 \end{pmatrix} = 3$

        $|A \cap B \cap C| = |\Set{b_A \geq 5 \land b_B \geq 5 \land b_C \geq 5}| = 0$

        Итого получаем $28 \cdot 3 + 3 \cdot 3 + 0 = 75$ вариантов.

        Далее исключаем эти варианты из количества всех вариантов, а значит подходящих вариантов всего $78 - 75 = 3$

        \vspace{5mm}
        \item \textbf{Принцип включения/исключения} (Inclusion/Exclusion Principle (PIE))

        \begin{itemize}
            \item $X$ - начальное множество элементов
            \item $P_1, \dots, P_m$ - свойства
            \item Пусть $X_i = \Set{x \in X \ | \ P_i\text{ - свойство для } x}$
            \item Пусть $S \in [m]$ - множество свойств
            \item Пусть $N(S) = \bigcap_{i \in S} X_i = \Set{x \in X\ | \ x \text{ имеет все свойства } P_1, \dots, P_m}$
        \end{itemize}

        $N(\emptyset) = X \quad |N(\emptyset)| = |X| = n$

        \vspace{5mm}
        \item \textbf{Теорема ПВ/И} (Theorem PIE)

        $|X \setminus (X_1 \union X_2 \union \dots \union X_m)| = \sum_{S \subseteq [m]} (-1)^{|S|} |N(S)|$ - количество элементов множества $X$, не имеющих никакое из свойств

        Доказательство:

        Пусть $x \in X$

        Если $x$ не имеет свойств $P_1,\dots,P_m$, то $x \in N(\emptyset)$ и $x \notin N(S) \ \forall S \neq \emptyset$

        Поэтому $x$ дает в общую сумму $1$

        Иначе, если $x$ имеет $k \geq 1$ свойств $T \in \begin{pmatrix} [m] \\ k\end{pmatrix}$,

        то $x \in N(S)$ тогда и только тогда, когда $S \subseteq T$.

        Поэтому $x$ дает в сумму $\sum_{S \subseteq T} (-1)^{|S|} = \sum_{i = 0}^k \begin{pmatrix} k \\ i \end{pmatrix} (-1)^i = 0$

        \vspace{5mm}
        \item \textbf{Следствие}

        $|\bigunion_{i \in [m]} X_i| = |X| - \sum_{S \subseteq [m]} (-1)^{|S|} |N(S)| = \sum_{S \subseteq [m], S \neq \emptyset} (-1)^{|S| - 1} |N(S)|$

        \vspace{5mm}
        \item \textbf{Приложения}:

        * Определяете \enquote{плохие} свойства $P_1, \dots, P_m$

        * Посчитываете $N(S)$

        * Применяете ПВ/И

        \vspace{5mm}
        \item \textbf{Количество сюръекций (правототальных функций)}

        * $X = \Set{\text{функция } f : [k] \to [n]}$

        * Плохое свойство $P_i \ : \ X_i = \Set{f : [k] \to [n] \ | \ \nexists j \in [k] : f(j) = i}$

        * $|\Set{\text{сюръекции } f : [k] \to [n]}| = |X \setminus (X_1 \union \dots \union X_m)| \stackrel{\text{PIE}}{=}
        \sum_{S \subseteq [m]} (-1)^{|S|} |N(S)| = \sum_{S \subseteq [m]} (-1)^{|S|} (n - |S|)^k =
        \sum^k_{i = 0} (-1)^{i} \begin{pmatrix} k \\ i \end{pmatrix} (k - i)^n$

        \vspace{5mm}
        \item \textbf{Количество биекций}

        $n! = \sum_{i=0}^n (-1)^i \begin{pmatrix}
                                      n \\ i
        \end{pmatrix} (n - i)^n$

        \item \textbf{Число Стирлинга} (опять)

        Заметим, что сюръекция = разбиение, тогда:

        $\sum^k_{i = 0} (-1)^{i} \begin{pmatrix} k \\ i \end{pmatrix} (k - i)^n = n! S^{II}_n (k)$

        \vspace{5mm}
        \item \textbf{Беспорядки} (Derangements) - перестановка без фиксированных точек

        Если $f(i) = i$, то $i$ - фиксированная точка

        * $X = $ все $n!$ перестановок

        * Плохие свойства $P_1,\dots,P_m : \pi \in X$ имеет свойство $P_i$ \Longleftrightarrow $\pi(i) = i$

        * Посчитаем $N(S): \quad N(S) = (n - |S|)!$

        * Применяем ПВ/И: $X \setminus (X_1 \union \dots \union X_n) = \sum_{S \subseteq [n]} (-1)^{|S|} N(S) =
        \sum_{S \subseteq [n]} (-1)^{|S|} (n - |S|)! = \sum_{i = 0}^n (-1)^{i} \begin{pmatrix}
                                                                                   n \\ i
        \end{pmatrix} (n - i)!$

    \end{itemize}

    \clearpage


    \section{8. Рекуррентности и производящие функции}

    \begin{itemize}
        \item \textbf{Производящие функции} (Generating Functions)

        $\sum_{n = 0}^\infty a_n x^n = a_0 + a_1 x + a_2 x^2 + \dots$

        Функция выше задает последовательность $a_0, a_1, a_2, \dots$

        \Ex $3 + 8x^2 + x^3 + \frac{1}{7}x^5 + 100x^6 + \dots \implies (3, 0, 8, 1, 0, \frac{1}{7}, 100, \dots)$

        \Ex Последовательность $(1, 1, 1, \dots)$ задает функцию $1 + x + x^2 + \dots = \sum_{n = 0}^\infty x^n$

        Пусть $S = 1 + x + x^2 + \dots$, тогда $xS = x + x^2 + \dots$, $(1 - x) S = 1 \Longrightarrow $

        \fbox{$S = \frac{1}{1 - x}$ задает последовательность $(1, 1, 1, \dots)$}

        \Ex

        $\frac{1}{1 + x} = 1 - x + x^2 - x^3 + \dots = \sum_{n = 0}^\infty (-1)^n x^n$

        $\frac{1}{1 - 3x} = 1 + 3x + 9x^2 + 27x^3 + \dots = \sum_{n = 0}^\infty 3^n x^n$

        $\frac{2}{1 - x} = 2 + 2x + 2x^2 + 2x^3 + \dots = \sum_{n = 0}^\infty 2 x^n$

        $(2, 4, 10, 28, 82, \dots) = (1, 1, 1, 1, 1, \dots) + (1, 3, 9, 27, 81, \dots)$

        $\frac{1}{1 - x} + \frac{1}{1 - 3x} = \frac{2 - 4x}{(1 - x)(1 - 3x)}$

        $\frac{1}{1 - x^2} = 1 + x^2 + x^4 + x^6 + \dots = \sum_{n = 0}^\infty x^{2n} \implies (1, 0, 1, 0, \dots)$

        $\frac{x}{1 - x^2} = x + x^3 + x^5 + \dots = \sum_{n = 0}^\infty x^{2n + 1} \implies (0, 1, 0, 1, \dots)$

        \textbf{Взятие производной}:

        $\frac{d}{dx} (\frac{1}{1 - x}) = \frac{1}{(1 - x)^2} = \frac{d}{dx} (1 + x + x^2 + \dots) = 1 + 2x + 3x^2 + 4x^3 + \dots \implies (1, 2, 3, 4, \dots)$

        \Ex Найти ПФ для $(1, 3, 5, 7, 9, \dots)$

        $A(x) = 1 + 3x + 5x^2 + \dots$

        $xA = 0 + x + 3x^2 + 5x^3 + \dots$

        $(1 - x)A = 1 + 2x + 2x^2 + 2x^3 + \dots$

        $(1 - x)A = 1 + \frac{2x}{1 - x} \quad A = \frac{1 + \frac{2x}{1 - x}}{1 - x} = \frac{1 + x}{(1 - x)^2}$

        \Ex Найти ПФ для $(1, 4, 9, 16, \dots)$

        $A = 1 + 4x + 9x^2 + 16x^3 + \dots \quad (1 - x)A = $

        \item \textbf{Подсчет, используя производящие функции}

        Найти число решений для $x_1 + x_2 + x_3 = 6$, где $x_i \geq 0, x_1 \leq 4, x_2 \leq 3, x_3 \leq 5$

        $A_1(x) = 1 + x + x^2 + x^3 + x^4$

        $A_2(x) = 1 + x + x^2 + x^3$

        $A_3(x) = 1 + x + x^2 + x^3 + x^4 + x^5$

        $A(x) = A_1 \cdot A_2 \cdot A_3 = 1 + 3x + 6x^2 + 10x^3 + 14x^4 + 17x^5 + \underline{18x^6} + 17x^7 + \dots$

        Ответ - 18

    \end{itemize}


\end{document}
