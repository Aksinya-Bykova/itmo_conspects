\documentclass[12pt]{article}
\usepackage{preamble}

\pagestyle{fancy}
\fancyhead[LO,LE]{$\mathcal{D}$искретная математика}
\fancyhead[RO,RE]{Лекции Чухарева К. И.}

\renewcommand{\thesection}{}

\begin{document}

    \tableofcontents
    \clearpage

    
    \section{7. Комбинаторика}

    \textbf{Базовые понятия:}

    \begin{itemize}
        \item \textbf{Алфавит} (Alphabet) $\Sigma$ (или $X$, \Exs $X = \Set{a, b, c}$) - множество символов в нашей системе

        \vspace{5mm}

        \item \textbf{Диапазон} (Range) $[n] = \Set{1, \dots, n}$ - конечное множество последовательных натуральных чисел

        \vspace{5mm}

        \item \textbf{Расстановка} (Ordered arrangement) - последовательность каких-либо элементов (тоже самое, что кортеж),
        \Exs $x = (a, b, c, d, b, b, c) \quad |x| = n$

        Расстановку можно представить как функцию $f : \underset{\text{domain}}{\undergroup{[n]}} \to \underset{\text{codomain}}{\undergroup{\Sigma}}$, которая по порядковому номеру выдает символ

        $ran f = \Set{c \in \Sigma \ | \ \exists i \in [n]\ :\ f(i) = c}$

        \vspace{5mm}

        \item \textbf{Перестановка} (Permutation) - $\pi : [n] \to \Sigma$, где $n = |\Sigma|$

        Расстановка $\pi$ - биекция между $[n]$ и $\Sigma$

        \Ex $\pi = \mathtt{2713546}$

        \vspace{3mm}

        \begin{tabular}{l|ccccccc}
            i      & 1          & 2          & 3          & 4          & 5          & 6          & 7          \\
            \hline
            \pi(i) & \mathtt{2} & \mathtt{7} & \mathtt{1} & \mathtt{3} & \mathtt{5} & \mathtt{4} & \mathtt{6}
        \end{tabular}

        \vspace{5mm}

        \underline{Одна из задач комбинаторики} - посчитать количество различных расстановок или перестановок при заданных $n$ и $\Sigma$}

        \vspace{5mm}

        \item \textbf{$k$-перестановка} (k-permutation) - расстановка из $k$ различных элементов из $\Sigma$

        \Ex $\underset{\text{5-perm из } \Sigma = [7]}{\undergroup{|31475|}} = 5$

        $k$-перестановка - инъекция $\pi : [k] \to \Sigma$ ($k \leq n = |\Sigma|$)

        \vspace{5mm}

        \item $P(n, k)$ - множество всех $k$-перестановок алфавита $\Sigma = [n]$ (если исходный алфавит не состоит из чисел, то мы можем сделать биекцию между ним и $[n]$)

        $P(n, k) = \Set{f \ | \ f : [k] \to [n]}$

        Чаще интересует не само множество, а его размер, поэтому под обозначением $P(n, k)$ подразумевается $|P(n, k)|$

        \vspace{5mm}

        \item $S_n = P_n = P(n, n)$ - множество всех перестановок. Также чаще всего нас будет интересовать не множество, а его размер

        $|S_n| = n!$ - всего существует $n!$ перестановок

        $|P(n, k)| = n \cdot (n - 1) \cdot (n - 2) \cdot \dots \cdot (n - k + 1) = \frac{n!}{(n - k)!}$

        \vspace{5mm}

        \item \textbf{Циклические $k$-перестановки} (Circular $k$-permutations)

        $\pi_1, \pi_2 \in P(n, k)$ - циклические эквивалентны тогда и только тогда:

        $\exists s \ | \ \forall i \ \pi_1((i + s) \% k) = \pi_2(i)$


        \Ex $\pi_1 = \mathtt{76123}, \pi_2 = \mathtt{12376}$

        \begin{tikzpicture}
        \node[circle, draw=black!60, thick, minimum size=0.5cm] (s00) {\mathtt{7}};
        \node[below=1cm of s00] (pi1) {$\pi_1$};
        \node[below left=0.66cm and 1cm of s00] (s01) {\mathtt{6}};
        \node[below right=1cm and 0.13cm of s01] (s02) {\mathtt{1}};
        \node[below right=0.66cm and 1cm of s00] (s04) {\mathtt{3}};
        \node[below left=1cm and 0.13cm of s04] (s03) {\mathtt{2}};
        \path[->]
        (s00) edge [bend right] node {} (s01)
        (s01) edge [bend right] node {} (s02)
        (s02) edge [bend right] node {} (s03)
        (s03) edge [bend right] node {} (s04)
        (s04) edge [bend right] node {} (s00)
        ;

        \node[circle, draw=black!60, thick, minimum size=0.5cm, right=3cm of s00] (s0) {\mathtt{1}};
        \node[below=1cm of s0] (pi) {$\pi_2$};
        \node[below left=0.66cm and 1cm of s0] (s1) {\mathtt{2}};
        \node[below right=1cm and 0.13cm of s1] (s2) {\mathtt{3}};
        \node[below right=0.66cm and 1cm of s0] (s4) {\mathtt{6}};
        \node[below left=1cm and 0.13cm of s4] (s3) {\mathtt{7}};
        \path[->]
        (s0) edge [bend right] node {} (s1)
        (s1) edge [bend right] node {} (s2)
        (s2) edge [bend right] node {} (s3)
        (s3) edge [bend right] node {} (s4)
        (s4) edge [bend right] node {} (s0)
        ;
        \end{tikzpicture}

        $P_C(n, k)$ - множество всех циклических $k$-перестановок в $\Sigma$

        $|P_C(n, k)| \cdot k = |P(n, k)|$

        $|P_C(n, k)| = \frac{|P(n, k)|}{k} = \frac{n!}{k(n - k)!}$

        \vspace{5mm}

        \item \textbf{Неупорядоченная расстановка $k$ элементов} (Unordered arrangement of $k$ elements) - мультимножество $\Sigma^*$ размера $k$

        \Ex $\Sigma^* = \Set{\triangle, \triangle, \Box, \triangle, \circ, \Box}^* = \Set{3 \cdot \triangle, 2 \cdot \Box, 1 \cdot \circ} = (\Sigma, r)$

        Неупорядоченную расстановку можно представить как функцию:

        $r : \Sigma \to \Natural, \quad r(x)$ - кол-во повторений объекта $x$

        \vspace{5mm}

        \item \textbf{$k$-сочетание} ($k$-combination) - неупорядоченная перестановка из $k$ различных элементов из $\Sigma$ (еще называют $k$-подмножеством, $k$-subset)

        Соответственно $C(n, k)$ - множество всех таких $k$-сочетаний

        $|C(n, k)| = C^k_n = \begin{pmatrix}n \\ k\end{pmatrix}$

        $C(n, k) = \begin{pmatrix}\Sigma \\ k\end{pmatrix}$

        $\begin{pmatrix}n \\ k\end{pmatrix} \cdot k! = |P(n, k)|$

        $|C(n, k)| = \begin{pmatrix}n \\ k\end{pmatrix} = \frac{n!}{k!(n - k)!}$

    \end{itemize}

    \Th Биномиальная теорема (Binomial theorem):

    \[(x + y)^n = \sum_{k=0}^n \begin{pmatrix}n \\ k\end{pmatrix} x^k y^{n - k}\]

    $\begin{pmatrix}n \\ k\end{pmatrix}$ - биномиальный коэффициент

    \Th Мультиномиальная теорема (Multinomial theorem)

    \[(x_1 + \dots + x_r)^n = \sum_{\substack{k_i \in 1..n, \\ k_1 + \dots + k_r = n}} \begin{pmatrix}n \\ k_1, \dots, k_r\end{pmatrix} x^{k_1}_1 \cdot \dots \cdot x^{k_r}_r\]

    $\begin{pmatrix}n \\ k_1, \dots, k_r\end{pmatrix} = \frac{n!}{k_1! \dots k_r!}$ - мультиномиальный коэффициент




\end{document}

