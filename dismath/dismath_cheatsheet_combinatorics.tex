\documentclass[12pt]{article}
\usepackage{preamble}

\pagestyle{fancy}
\pagenumbering{gobble}

\fancyhead[LO,LE]{Шпаргалка по \\ Комбинаторике {\LARGE 🎲}}
\fancyhead[RO,RE]{$\mathcal{D}$искретная математика \\ Лекции Чухарева К. И.}

\fancyfoot[L]{\scriptsize исходники есть тута: \url{https://github.com/pelmesh619/itmo_conspects} \\ made by discrete math lovers \Cat}

\begin{document}
    \section{7 \quad Комбинаторика}

    \begin{itemize}
        \item \textbf{Алфавит} $\Sigma$ (или $X$, \Exs $X = \Set{a, b, c}$) - множество символов

        \item \textbf{Диапазон} $[n] = \Set{1, \dots, n}$ - конечное множество последовательных натуральных чисел

        \item \textbf{Расстановка} - последовательность каких-либо элементов (кортеж)
        \hfill\href{https://ru.wikipedia.org/wiki/%D0%9A%D0%BE%D1%80%D1%82%D0%B5%D0%B6_(%D0%B8%D0%BD%D1%84%D0%BE%D1%80%D0%BC%D0%B0%D1%82%D0%B8%D0%BA%D0%B0)#%D0%92_%D0%BC%D0%B0%D1%82%D0%B5%D0%BC%D0%B0%D1%82%D0%B8%D0%BA%D0%B5}{*тык*}

        \Exs $x = (a, b, c, d, b, b, c) \quad |x| = n$

        Расстановку можно представить как функцию $f : [n] \to \Sigma$

        \item \textbf{Перестановка} - $\pi : [n] \to \Sigma$, где $n = |\Sigma|$
        \hfill\href{https://ru.wikipedia.org/wiki/%D0%9F%D0%B5%D1%80%D0%B5%D1%81%D1%82%D0%B0%D0%BD%D0%BE%D0%B2%D0%BA%D0%B0}{*тык*}

        Расстановка $\pi$ - биекция между $[n]$ и $\Sigma$

        \Exs $\pi = \mathtt{2713546}$

        \vspace{3mm}

        \begin{tabular}{l|ccccccc}
            i      & 1          & 2          & 3          & 4          & 5          & 6          & 7          \\
            \hline
            \pi(i) & \mathtt{2} & \mathtt{7} & \mathtt{1} & \mathtt{3} & \mathtt{5} & \mathtt{4} & \mathtt{6}
        \end{tabular}

        \item \textbf{$k$-перестановка} - расстановка из $k$ различных элементов из $\Sigma$

        \Exs $5$-перестановка из $\Sigma = [7]$ - $|31475| = 5$

        $k$-перестановка - это инъекция $\pi : [k] \to \Sigma$ ($k \leq n = |\Sigma|$)

        \item $P(n, k)$ - множество всех $k$-перестановок алфавита $\Sigma = [n]$ (если исходный алфавит не состоит из чисел, то мы можем сделать биекцию между ним и $[n]$)

        $P(n, k) = \Set{f \ | \ f : [k] \to [n]}$
        
        \item $S_n = P_n = P(n, n)$ - множество всех перестановок. 

        $|S_n| = n!$ - всего существует $n!$ перестановок

        \fbox{$|P(n, k)| = n \cdot (n - 1) \cdot (n - 2) \cdot \dots \cdot (n - k + 1) = \frac{n!}{(n - k)!}$}

        \item \textbf{Циклические $k$-перестановки}
        \hfill\href{https://ru.wikipedia.org/wiki/%D0%A6%D0%B8%D0%BA%D0%BB%D0%B8%D1%87%D0%B5%D1%81%D0%BA%D0%B0%D1%8F_%D0%BF%D0%B5%D1%80%D0%B5%D1%81%D1%82%D0%B0%D0%BD%D0%BE%D0%B2%D0%BA%D0%B0}{*тык*}

        \begin{minipage}{\linewidth}
            \begin{wrapfigure}{r}{0pt}
                \begin{tikzpicture}
                \node[circle, draw=black!60, thick, minimum size=0.2cm] (s00) {\mathtt{7}};
                \node[below=0.4cm of s00] (pi1) {$\pi_1$};
                \node[below left=0.26cm and 0.4cm of s00] (s01) {\mathtt{6}};
                \node[below right=0.4cm and 0.02cm of s01] (s02) {\mathtt{1}};
                \node[below right=0.26cm and 0.4cm of s00] (s04) {\mathtt{3}};
                \node[below left=0.4cm and 0.02cm of s04] (s03) {\mathtt{2}};
                \path[->]
                (s00) edge [bend right] node {} (s01)
                (s01) edge [bend right] node {} (s02)
                (s02) edge [bend right] node {} (s03)
                (s03) edge [bend right] node {} (s04)
                (s04) edge [bend right] node {} (s00)
                ;

                \node[circle, draw=black!60, thick, minimum size=0.2cm, right=20mm of s00] (s10) {\mathtt{1}};
                \node[below=0.4cm of s10] (pi2) {$\pi_2$};
                \node[below left=0.26cm and 0.4cm of s10] (s11) {\mathtt{2}};
                \node[below right=0.4cm and 0.02cm of s11] (s12) {\mathtt{3}};
                \node[below right=0.26cm and 0.4cm of s10] (s14) {\mathtt{6}};
                \node[below left=0.4cm and 0.02cm of s14] (s13) {\mathtt{7}};
                \path[->]
                (s10) edge [bend right] node {} (s11)
                (s11) edge [bend right] node {} (s12)
                (s12) edge [bend right] node {} (s13)
                (s13) edge [bend right] node {} (s14)
                (s14) edge [bend right] node {} (s10)
                ;

                \node[below right=20mm and -14mm of s00] (label) {\Exs $\pi_1 = \mathtt{76123}, \pi_2 = \mathtt{12376}$}
                \end{tikzpicture}
            \end{wrapfigure}


            $\pi_1, \pi_2 \in P(n, k)$ - циклически эквивалентны тогда и только тогда:

            \fbox{$\exists s \ | \ \forall i \ \pi_1((i + s) \% k) = \pi_2(i)$}

            $P_C(n, k)$ - множество всех циклических $k$-перестановок в $\Sigma$

            $|P_C(n, k)| \cdot k = |P(n, k)|$

            \fbox{$|P_C(n, k)| = \frac{|P(n, k)|}{k} = \frac{n!}{k(n - k)!}$}

        \end{minipage}


        \item \textbf{Неупорядоченная расстановка $k$ элементов} - мультимножество $\Sigma^*$ размера $k$

        \Exs $\Sigma^* = \Set{\triangle, \triangle, \Box, \triangle, \circ, \Box}^* = \Set{3 \cdot \triangle, 2 \cdot \Box, 1 \cdot \circ} = (\Sigma, r)$

        Неупорядоченную расстановку можно представить как функцию:

        $r : \Sigma \to \Natural, \quad r(x)$ - кол-во повторений объекта $x$


        \item \textbf{$k$-сочетание} - неупорядоченная перестановка из $k$ различных элементов из $\Sigma$ (еще называют $k$-подмножеством, $k$-subset)
        \hfill\href{https://ru.wikipedia.org/wiki/%D0%A1%D0%BE%D1%87%D0%B5%D1%82%D0%B0%D0%BD%D0%B8%D0%B5}{*тык*}

        Соответственно $C(n, k)$ - множество всех таких $k$-сочетаний

        \fbox{$|C(n, k)| = C^k_n = \begin{pmatrix}n \\ k\end{pmatrix} = \frac{n!}{k!(n - k)!}$}
        $\quad\quad |P(n, k)| = C(n, k) \cdot k! = \begin{pmatrix}n \\ k\end{pmatrix} \cdot k!$

        \item \Ths \textbf{Биномиальная теорема}:

        \[(x + y)^n = \sum_{k=0}^n \begin{pmatrix}n \\ k\end{pmatrix} x^k y^{n - k}\]

        где $\begin{pmatrix}n \\ k\end{pmatrix}$ - биномиальный коэффициент
        \hfill\href{https://ru.wikipedia.org/wiki/%D0%91%D0%B8%D0%BD%D0%BE%D0%BC%D0%B8%D0%B0%D0%BB%D1%8C%D0%BD%D1%8B%D0%B9_%D0%BA%D0%BE%D1%8D%D1%84%D1%84%D0%B8%D1%86%D0%B8%D0%B5%D0%BD%D1%82}{*тык*}

        \item \Ths \textbf{Мультиномиальная теорема}:

        \[(x_1 + \dots + x_r)^n = \sum_{\substack{k_i \in 1..n, \\ k_1 + \dots + k_r = n}} \begin{pmatrix}n \\ k_1, \dots, k_r\end{pmatrix} x^{k_1}_1 \cdot \dots \cdot x^{k_r}_r\]

        где $\begin{pmatrix}n \\ k_1, \dots, k_r\end{pmatrix} = \frac{n!}{k_1! \dots k_r!}$ - мультиномиальный коэффициент
        \hfill\href{https://ru.wikipedia.org/wiki/%D0%9C%D1%83%D0%BB%D1%8C%D1%82%D0%B8%D0%BD%D0%BE%D0%BC%D0%B8%D0%B0%D0%BB%D1%8C%D0%BD%D1%8B%D0%B9_%D0%BA%D0%BE%D1%8D%D1%84%D1%84%D0%B8%D1%86%D0%B8%D0%B5%D0%BD%D1%82}{*тык*}


        \item \textbf{Перестановка мультимножества $\Sigma^*$}
        \hfill\href{https://ru.wikipedia.org/wiki/%D0%9F%D0%B5%D1%80%D0%B5%D1%81%D1%82%D0%B0%D0%BD%D0%BE%D0%B2%D0%BA%D0%B0#%D0%9F%D0%B5%D1%80%D0%B5%D1%81%D1%82%D0%B0%D0%BD%D0%BE%D0%B2%D0%BA%D0%B8_%D1%81_%D0%BF%D0%BE%D0%B2%D1%82%D0%BE%D1%80%D0%B5%D0%BD%D0%B8%D0%B5%D0%BC}{*тык*}

        $\Sigma^* = \Set{\triangle^1, \triangle^2, \Box, \star} = (\Sigma, r) \quad r : \Sigma \to \Natural_0 \quad n = |\Sigma^*| = 4 \quad s = |\Sigma| = 3$

        \Notas \begin{cases}
                  \triangle^1, \triangle^2, \Box, \star \\
                  \triangle^2, \triangle^1, \Box, \star
        \end{cases} считаются равными перестановками

        $|P^*(\Sigma^*, n)| = \frac{n!}{r_1! \dots r_s!} = \begin{pmatrix}
                                                               n \\ r_1, \dots, r_s
        \end{pmatrix}$ - количество перестановок мультимножества, где $r_i$ - количество $i$-ого элемента в мультимножестве

        \item \textbf{$k$-сочетание бесконечного мультимножества} -
        такое подмультимножество размера $k$, содержащее элементы из исходного мультимножества $\Sigma^*$.
        При этом соблюдается, что количество какого-либо элемента $r_i$ в исходном мультимножестве не больше размера сочетания $k$
        \hfill\href{https://ru.wikipedia.org/wiki/%D0%A1%D0%BE%D1%87%D0%B5%D1%82%D0%B0%D0%BD%D0%B8%D0%B5#%D0%A1%D0%BE%D1%87%D0%B5%D1%82%D0%B0%D0%BD%D0%B8%D1%8F_%D1%81_%D0%BF%D0%BE%D0%B2%D1%82%D0%BE%D1%80%D0%B5%D0%BD%D0%B8%D1%8F%D0%BC%D0%B8}{*тык*}


        \fbox{$\frac{(k + s - 1)!}{k!(s - 1)!} = \begin{pmatrix} k + s - 1 \\ k, s - 1 \end{pmatrix} =
        \begin{pmatrix} k + s - 1 \\ k \end{pmatrix} = \begin{pmatrix} k + s - 1 \\ s - 1 \end{pmatrix}$,}

        где $k$ - размер сочетания, $s = |\Sigma|$ - количество уникальных элементов в множестве


        \item \textbf{Слабая композиция} неотрицательного целого числа $n$ в $k$ частей -
        это решение $(b_1, \dots, b_k)$ уравнение $b_1 + \dots + b_k = n$, где $b_i \geq 0$

        $|\Set{\text{слабая композиция } n \text{ в } k \text{ частей}}| = \begin{pmatrix}
                                                                               n + k - 1 \\ n, k - 1
        \end{pmatrix}$


        \item \textbf{Композиция} - решение для $b_1 + \dots + b_k = n$, где $b_i > 0$

        $|\Set{\text{композиция } n \text{ в } k \text{ частей}}| =
        \begin{pmatrix} n - k + k - 1 \\ n - k, k - 1 \end{pmatrix}$

        \item \textbf{Число всех композиций $n$ в некоторой число частей}:

        \[\sum_{k=1}^n \begin{pmatrix}
                          n - 1 \\ k - 1
        \end{pmatrix} = 2^{n-1}\]


        \item \textbf{Разбиения множества} - множество размера $k$ непересекающихся непустых подмножеств
        \hfill\href{https://ru.wikipedia.org/wiki/%D0%A0%D0%B0%D0%B7%D0%B1%D0%B8%D0%B5%D0%BD%D0%B8%D0%B5_%D0%BC%D0%BD%D0%BE%D0%B6%D0%B5%D1%81%D1%82%D0%B2%D0%B0#%D0%A0%D0%B0%D0%B7%D0%B1%D0%B8%D0%B5%D0%BD%D0%B8%D1%8F_%D0%BA%D0%BE%D0%BD%D0%B5%D1%87%D0%BD%D1%8B%D1%85_%D0%BC%D0%BD%D0%BE%D0%B6%D0%B5%D1%81%D1%82%D0%B2}{*тык*}

        $|\Set{\text{разбиение } n \text{ элементов в } k \text{ частей}}| =
        \begin{Bmatrix} n \\ k \end{Bmatrix} = S^{II}_k (n) = S(n, k)$ - число Стирлинга второго рода \hfill\href{https://ru.wikipedia.org/wiki/%D0%A7%D0%B8%D1%81%D0%BB%D0%B0_%D0%A1%D1%82%D0%B8%D1%80%D0%BB%D0%B8%D0%BD%D0%B3%D0%B0_%D0%B2%D1%82%D0%BE%D1%80%D0%BE%D0%B3%D0%BE_%D1%80%D0%BE%D0%B4%D0%B0}{*тык*}


        \item \textbf{Формула Паскаля}:
        \hfill\href{https://ru.wikipedia.org/wiki/%D0%91%D0%B8%D0%BD%D0%BE%D0%BC%D0%B8%D0%B0%D0%BB%D1%8C%D0%BD%D1%8B%D0%B9_%D0%BA%D0%BE%D1%8D%D1%84%D1%84%D0%B8%D1%86%D0%B8%D0%B5%D0%BD%D1%82#%D0%A2%D1%80%D0%B5%D1%83%D0%B3%D0%BE%D0%BB%D1%8C%D0%BD%D0%B8%D0%BA_%D0%9F%D0%B0%D1%81%D0%BA%D0%B0%D0%BB%D1%8F}{*тык*}

        \[\begin{pmatrix}
             n \\ k
        \end{pmatrix} = \begin{pmatrix}
                            n - 1 \\ k - 1
        \end{pmatrix} + \begin{pmatrix}
                            n - 1 \\ k
        \end{pmatrix}\]

        
        \item \textbf{Рекуррентное отношение для чисел Стирлинга}:

        \[\begin{Bmatrix}
             n \\ k
        \end{Bmatrix} = \begin{Bmatrix}
                            n - 1 \\ k - 1
        \end{Bmatrix} + k \cdot \begin{Bmatrix}
                                    n - 1 \\ k
        \end{Bmatrix}\]

        \item \textbf{Число Белла} - количество всех неупорядоченных разбиений множества размера $n$
        \hfill\href{https://ru.wikipedia.org/wiki/%D0%A7%D0%B8%D1%81%D0%BB%D0%BE_%D0%91%D0%B5%D0%BB%D0%BB%D0%B0}{*тык*}

        Число Белла вычисляется по формуле: $B_n = \sum_{m=0}^n S(n, m)$

        
        \item \textbf{Целочисленное разбиение} - решение для $a_1 + \dots + a_k = n$, где $a_1 \geq a_2 \geq \dots \geq a_k \geq 1$

        $p(n, k)$ - число целочисленных разбиений $n$ в $k$ частей

        $p(n) = \sum_{k = 1}^n p(n, k)$ - число всех разбиений для $n$

        \Exs $5 = 5 = 4 + 1 = 3 + 2 = 3 + 1 + 1 = 2 + 2 + 1 = 2 + 1 + 1 + 1 = 1 + 1 + 1 + 1 + 1$


        \item \textbf{Принцип включений/исключений}:
        \hfill\href{https://ru.wikipedia.org/wiki/%D0%A4%D0%BE%D1%80%D0%BC%D1%83%D0%BB%D0%B0_%D0%B2%D0%BA%D0%BB%D1%8E%D1%87%D0%B5%D0%BD%D0%B8%D0%B9-%D0%B8%D1%81%D0%BA%D0%BB%D1%8E%D1%87%D0%B5%D0%BD%D0%B8%D0%B9}{*тык*}

        \begin{itemize}
            \item $X$ - начальное множество элементов
            \item $P_1, \dots, P_m$ - свойства
            \item Пусть $X_i = \Set{x \in X \ | \ P_i\text{ - свойство для } x}$
            \item Пусть $S \in [m]$ - множество свойств
            \item Пусть $N(S) = \bigcap_{i \in S} X_i = \Set{x \in X\ | \ x \text{ имеет все свойства } P_1, \dots, P_m}$
        \end{itemize}

        \Exs $N(\emptyset) = X \quad |N(\emptyset)| = |X| = n$

        \item \textbf{Формула включений/исключений}:
        \hfill\href{https://ru.wikipedia.org/wiki/%D0%A4%D0%BE%D1%80%D0%BC%D1%83%D0%BB%D0%B0_%D0%B2%D0%BA%D0%BB%D1%8E%D1%87%D0%B5%D0%BD%D0%B8%D0%B9-%D0%B8%D1%81%D0%BA%D0%BB%D1%8E%D1%87%D0%B5%D0%BD%D0%B8%D0%B9#%D0%92_%D1%82%D0%B5%D1%80%D0%BC%D0%B8%D0%BD%D0%B0%D1%85_%D1%81%D0%B2%D0%BE%D0%B9%D1%81%D1%82%D0%B2}{*тык*}

        $|X \setminus (X_1 \union X_2 \union \dots \union X_m)| = \sum_{S \subseteq [m]} (-1)^{|S|} |N(S)|$ - количество элементов множества $X$, не имеющих никакое из свойств

        \item \textbf{Следствие}:

        $|\bigunion_{i \in [m]} X_i| = |X| - \sum_{S \subseteq [m]} (-1)^{|S|} |N(S)| = \sum_{S \subseteq [m], S \neq \emptyset} (-1)^{|S| - 1} |N(S)|$

        
        \item \textbf{Приложения}:

        * Определяем \enquote{плохие} свойства $P_1, \dots, P_m$

        * Посчитываем $N(S)$

        * Применяем ПВ/И

        \item \textbf{Количество сюръекций (правототальных функций)}:

        * $X = \Set{\text{функция } f : [k] \to [n]}$

        * Плохое свойство $P_i \ : \ X_i = \Set{f : [k] \to [n] \ | \ \nexists j \in [k] : f(j) = i}$

        * $|\Set{\text{сюръекции } f : [k] \to [n]}| = |X \setminus (X_1 \union \dots \union X_m)| =
        \sum_{S \subseteq [m]} (-1)^{|S|} |N(S)| = \\ \sum_{S \subseteq [m]} (-1)^{|S|} (n - |S|)^k = $
        \fbox{$\sum^k_{i = 0} (-1)^{i} \begin{pmatrix} k \\ i \end{pmatrix} (k - i)^n$}$ = S^{(II)}_n(k)$ - число Стирлинга второго рода

        
        \item \textbf{Количество биекций}:

        \[n! = \sum_{i=0}^n (-1)^i \begin{pmatrix}
                                      n \\ i
        \end{pmatrix} (n - i)^n\]

        \item \textbf{Беспорядки} - перестановка без фиксированных точек
        \hfill\href{https://ru.wikipedia.org/wiki/%D0%91%D0%B5%D1%81%D0%BF%D0%BE%D1%80%D1%8F%D0%B4%D0%BE%D0%BA_(%D0%BF%D0%B5%D1%80%D0%B5%D1%81%D1%82%D0%B0%D0%BD%D0%BE%D0%B2%D0%BA%D0%B0)}{*тык*}

        Если $f(i) = i$, то $i$ - фиксированная точка

        * $X = $ все $n!$ перестановок

        * Плохие свойства $P_1,\dots,P_m : \pi \in X$ имеет свойство $P_i$ \Longleftrightarrow $\pi(i) = i$

        * Посчитаем $N(S): \quad N(S) = (n - |S|)!$

        * Применяем ПВ/И: $X \setminus (X_1 \union \dots \union X_n) = \sum_{S \subseteq [n]} (-1)^{|S|} N(S) =
        \sum_{S \subseteq [n]} (-1)^{|S|} (n - |S|)! = \sum_{i = 0}^n (-1)^{i} \begin{pmatrix}
                                                                                   n \\ i
        \end{pmatrix} (n - i)!$

    \end{itemize}

\end{document}
