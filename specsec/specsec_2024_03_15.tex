\documentclass[12pt]{article}
\usepackage{preamble}

\pagestyle{fancy}
\fancyhead[LO,LE]{Специальные разделы \\ высшей математики}
\fancyhead[CO,CE]{15.03.2024}
\fancyhead[RO,RE]{Лекции Далевской О. П.}


\begin{document}
    \Def Произведение операторов (композиция)

    $\mathcal{A}\mathcal{B}$ - произведение, $\mathcal{A} : V \rightarrow W; \ \mathcal{B} : U \rightarrow V$

    $(\mathcal{A}\mathcal{B}) x = \mathcal{A}(\mathcal{B}x); \quad x \in U$

    \hypertarget{linearoperatorproperties}{}

    Свойства: \underline{Lab} доказать

    1* $\lambda (\mathcal{A}\mathcal{B}) = (\lambda \mathcal{A})\mathcal{B}$

    2* $(\mathcal{A} + \mathcal{B}) \mathcal{C} = \mathcal{A}\mathcal{C} + \mathcal{B}\mathcal{C}$

    3* $\mathcal{A} (\mathcal{B} + \mathcal{C}) = \mathcal{A}\mathcal{B} + \mathcal{A}\mathcal{C}$

    4* $\mathcal{A} (\mathcal{B}\mathcal{C}) = (\mathcal{A}\mathcal{B}) \mathcal{C}$


    \Nota Можно обобщить 4* на $n$ равных $\mathcal{A}$

    \Def $\mathcal{A}^n = \mathcal{A} \cdot \mathcal{A} \dots \mathcal{A}$ - $n$ раз, степень оператора

    Свойства: $\mathcal{A}^{m + n} = \mathcal{A}^n \cdot \mathcal{A}^m$

    \section[p2\_3]{2.3. Обратимость оператора}

    \hypertarget{onetoonelinearoperator}{}

    Def: $\mathcal{A} : V \rightarrow W$ так, что $\mathcal{A}V = W$ и $\forall x_1 \neq x_2 (x_1, x_2 \in V) \quad
    \begin{cases}y_1 = \mathcal{A}x_1 \\ y_2 = \mathcal{A}x_2\end{cases} \Longrightarrow y_1 \neq y_2$

    Тогда $\mathcal{A}$ называется взаимно-однозначно действующим

    Nota: Проще сказать \enquote{линейный изоморфизм}

    \Th $\Set{x_i}$ - линейно независима $\stackrel{\mathcal{A}x = y}{\longrightarrow} \Set{y_i}$ - линейно независима

    В обратную сторону, если $\mathcal{A}$ - взаимно-однозначен

    $\Box \sqsupset \mathcal{A} : V \rightarrow W$ и $\texttt{0}_V, \texttt{0}_W$ - нули $V$ и $W$ соответственно
    \begin{enumerate}
        \item $\mathcal{A}(\texttt{0}_V) = \mathcal{A}(\sum^k_{i=1} 0 \cdot e_i) = \sum^k_{i=1} 0 \cdot \mathcal{A}e_i = \texttt{0}_W$

        \item Докажем, что если ${x_i} \subset V$ - лин. нез., то ${y_i} \subset W$ - лин. нез.

        Составим $\sum^m_{j=1} \lambda_j y_j = \texttt{0}_W$ (От противного) $\sqsupset \Set{y_i}$ - лин. зав., тогда $\exists \lambda_k \neq 0$

        При этом $\forall j \ \ y_j = \mathcal{A}x_j$ (т. к. $\mathcal{A}$ - вз.-однозн., то $n^\prime = m^\prime$: кол-во $x_i$ и $y_i$ равно)

        $\sum^{m^\prime}_{j=1} \lambda_j \mathcal{A}x_j \stackrel{\text{линейность}}{=} \mathcal{A} (\sum^{m^\prime}_{j=1} \lambda_j x_j) = \texttt{0}_W$

        Так как $\mathcal{A}\texttt{0}_V = \texttt{0}_W$, то $\texttt{0}_W$ - образ $x = \texttt{0}_V$, но так как $\mathcal{A}$ - вз.-однозн., то
        $\nexists x^\prime \neq x \ | \ \mathcal{A}(x^\prime) = \texttt{0}_W$

        Значит $\sum^{m^\prime}_{j=1} \lambda_j x_j = \texttt{0}_V$, но $\exists \lambda_k \neq 0 \Longrightarrow \Set{x_j}$ - лин. зав. - \underline{противоречие}

        \item $\sqsupset$ теперь $\Set{y_i}$ - л. нез., а $\Set{x_i}$ (по предположению от противного) - лин. зав.

        $\sum^{n^\prime}_{i = 1} \lambda_i x_i \stackrel{\exists \lambda_k \neq 0}{=} \texttt{0}_V \quad \Big| \mathcal{A}$

        $\sum^{n^\prime}_{i = 1} \lambda_i \mathcal{A}x_i = \texttt{0}_W$

        При этом $\exists \lambda_k \neq 0 \Longrightarrow \Set{y_i}$ - лин. зав. - \underline{противоречие}

    \end{enumerate}

    Следствие: $\dim V = \dim W \Longleftarrow \mathcal{A}$ - лин. изоморфизм

    \hypertarget{reverselinearoperator}{}

    Def: $\mathcal{B} : W \rightarrow V$ называется обратным оператором для $\mathcal{A} : V \rightarrow W$

    если $\mathcal{B}\mathcal{A} = \mathcal{A}\mathcal{B} = \mathcal{I}$ (обозначается $\mathcal{B} = \mathcal{A}^{-1}$)

    Следствие: $\mathcal{A}\mathcal{A}^{-1} x = x$

    \Th $\mathcal{A}x = \texttt{0}$ и $\exists \mathcal{A}^{-1}$, тогда $x = \texttt{0}$

    $\Box \mathcal{A}^{-1}\mathcal{A} x = \mathcal{A}^{-1}(\mathcal{A} x) = \mathcal{A}^{-1} \texttt{0}_W = \texttt{0}_V \Longrightarrow x = \texttt{0}$

    \Th Необходимые и Достаточные условия существования $\mathcal{A}^{-1}$

    $\exists \mathcal{A}^{-1} \Longleftrightarrow \mathcal{A}$ - вз.-однозн.

    $\Box \Longrightarrow \exists \mathcal{A}^{-1}$, но $\sqsupset \mathcal{A}$ - не вз.-однозн., то есть
    $\exists x_1, x_2 \in V (x_1 \neq x_2) \ | \ \mathcal{A}x_1 = \mathcal{A}x_2 \Longleftrightarrow \mathcal{A}x_1 - \mathcal{A}x_2 = \texttt{0} \Longleftrightarrow
    \mathcal{A}(x_1 - x_2) = \texttt{0}_W \stackrel{\exists \mathcal{A}^{-1}}{\Longrightarrow} x = \texttt{0}_V \Longleftrightarrow x_1 = x_2$ - противоречие

    $\Longleftarrow$ Так как $\mathcal{A}$ - изоморфизм (не учитывая линейность), то $\exists \mathcal{A}^\prime$ - обратное отображение (не обязат. линейное)

    Докажем, что $\mathcal{A}^\prime : W \rightarrow V$ - линейный оператор

    ? $\mathcal{A}^\prime (\sum \lambda_i y_i) = \sum \lambda_i \mathcal{A}^\prime y_i = \sum \lambda_i x_i$

    $\mathcal{A}$ - вз.-однозн. $\Longleftrightarrow \forall x_i \longleftrightarrow y_i \quad \Big| \cdot \lambda_i, \sum$

    $\mathcal{A}(\sum \lambda_i x_i) = \mathcal{A} x = y = \sum \lambda_i y_i \quad$ и $y$ имеет только один прообраз $x$

    Применим $\mathcal{A}^\prime$ к $y = \sum \lambda_i y_i \quad \mathcal{A}^\prime y = x = \sum \lambda_i x_i$ - единственный прообраз $y$

    Таким образом, $\mathcal{A}^\prime$ переводит лин. комбинацию в такую же лин. комбинацию прообразов, то есть $\mathcal{A}^\prime$ - линейный: $\mathcal{A}^\prime = \mathcal{A}^{-1}$

    \section[p2\_4]{2.4. Матрица ЛО}

    $\mathcal{A} : V^n \rightarrow W^m$

    Возьмем вектор $x \in V^n$ и разложим по какому-либо базису $\Set{e_j}^n_{j=1}$

    $\mathcal{A}x = \mathcal{A} (\sum^n_{j=1} c_j e_j) = \sum^n_{j=1} c_j \mathcal{A}e_j$

    $\mathcal{A} e_j \stackrel{\text{образ базисного вектора}}{=} y_j \stackrel{\Set{f_i} - \text{ базис } W^m}{=} \sum^m_{i=1} a_{ij}f_i$

    $\mathcal{A}x = \sum^n_{j=1} c_j \mathcal{A}e_j = \sum^n_{j=1} c_j \sum^m_{i=1} a_{ij}f_i = \sum^n_{j=1} \sum^m_{i=1} c_j a_{ij} f_i = \sum^m_{i=1} \sum^n_{j=1} c_j a_{ij} f_i$

    Иллюстрация:

    $\begin{pmatrix}
         a_{11} & \dots & a_{1n} \\
         \vdots & \ddots & \vdots \\
         a_{m1} & \dots & a_{mn} \\
    \end{pmatrix} \begin{pmatrix}
         c_{1} \\
         \vdots \\
         c_{n} \\
    \end{pmatrix} = \begin{pmatrix}
         b_{1} \\
         \vdots \\
         b_{m} \\
    \end{pmatrix}$

    \hypertarget{operatorsmatrix}{}

    Def: Матрица $A = {a_{ij}}_{i=1..m, j=1..n}$ называется матрицей оператора $\mathcal{A} : V^n \rightarrow W^m$ в базисе $\Set{e_j}^n_{j=1}$ пространства $V^n$

    Вопросы:

    1) $\forall ? \mathcal{A} \ \exists A$

    2) $\forall ? A \ \exists \mathcal{A}$

    3) если $\exists A$ для $\mathcal{A}$, то единственная?

    4) если $\exists \mathcal{A}$ для $A$, то единственная?

    Ответы:

    1) При выбранном базисе $\Set{e_j} \ \forall \mathcal{A} \ \exists A$ (алгоритм выше)

    3) такая $A$ единственная $\Longrightarrow$ в разных базисах матрицы ЛО $\mathcal{A} \quad A_e \neq A_{e^\prime}$

    2) $\forall A_{m\times n}$ можно взять пару ЛП $V^n, W^m$ и определить $\mathcal{A} : V^n \rightarrow W_n$ по правилу $\mathcal{A}e_V = e_W^\prime$

    4) \Lab

    Nota: Далее будем решать две задачи

    1) преобразование координат как действие оператора

    2) поиск наиболее простой матрицы в некотором базисе

    \section[p2\_5]{2.5. Ядро и образ оператора}

    \hypertarget{kernalandimageofoperator}{}

    \Def Ядро оператора - $Ker \mathcal{A} \stackrel{def}{=} \Set{x \in V \ | \ \mathcal{A}x = \texttt{0}_W}$

    \Def Образ оператора - $Im \mathcal{A} \stackrel{def}{=} \Set{y \in W \ | \ \mathcal{A}x = y}$

    \Nota $Ker \mathcal{A}$ и $Im \mathcal{A}$ - подпространства


\end{document}


