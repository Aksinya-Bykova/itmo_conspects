\documentclass[12pt]{article}
\usepackage{preamble}

\pagestyle{fancy}
\fancyhead[LO,LE]{Физические основы компьютерных \\ и сетевых технологий}
\fancyhead[CO,CE]{16.09.2024}
\fancyhead[RO,RE]{Лекции Музыченко Я. Б.}

\fancyfoot[L]{\scriptsize исходники найдутся тут: \\ \url{https://github.com/pelmesh619/itmo_conspects} \Cat}

\begin{document}
    \section{3.
    Кинематика вращательного движения.
    Динамика материальной точки}

    \begin{tcolorbox}[colframe=blue!25, colback=blue!10, title=\textbf{План лекции}]

        \footnotesize
        \begin{itemize}
            \item Угловые величины: угол поворота, угловая скорость

            \item Взаимосвязь между линейными и угловыми величинами

            \item Плоское движение

            \item Динамика материальной точки

            \item Законы Ньютона. Силы в механике

            \item Принципы работы акселерометра
        \end{itemize}
    \end{tcolorbox}

    \subsection{Движение по окружности}

    Возьмем точку $A$, положение которое определим через $\vec{r}$. Точка $A$ движется по окружности вокруг неподвижной оси $OO^\prime$

    Тогда $d\vec{r}$ - перемещение, $d\vec{\varphi}$ - элементарный угол поворота (вектор определяет в какую сторону, по часовой или против,
    обращается по окружности тело; вектор направлен перпендикулярно окружности)

    $|d\vec{r}| = Rd\varphi = r \cdot \sin \alpha d \varphi$

    $R = r \cdot \sin \alpha$

    $d\vec{r} = [d \vec{\varphi} \vec{r}]$ \hfill {\scriptsize здесь и далее $[\vec{x}\vec{y}]$ - векторное произведение}

    Угловая скорость - векторная величина, показывающая как меняется угол поворота тела со временем: $\Pair{\omega} = \frac{\Delta \varphi}{\Delta t} \quad\quad\quad \vec{\omega} = \frac{d\vec{\varphi}}{dt}$

    Направление совпадает с направлением угла поворота $d\vec{\varphi}$: $\vec{\omega} \uparrow\uparrow d\vec{\varphi}$

    Угловое ускорение - векторная величина, показывающая как меняется угловая скорость тела со временем

    $\Pair{\beta} = \frac{\Delta \omega}{\Delta t} \quad\quad \vec{\beta} = \frac{d\vec{\omega}}{dt} = \frac{d^2 \vec{\varphi}}{dt^2}$

    Направление совпадает с направлением вектора изменения скорости $\Delta \vec{\omega}$: $\vec{\beta} \uparrow\uparrow d\vec{\omega}$

    $d\vec{r} = [d\vec{\varphi} \vec{r}]$

    $dr = d\varphi \cdot r \cdot \sin \alpha = d\varphi \cdot R$

    Выразим скорость $\vec{v} = \frac{d\vec{r}}{dt} = [\frac{d\vec{\varphi}}{dt}\vec{r}] = [\vec{\omega} \vec{r}]$

    $v = \omega \cdot r \cdot \sin \alpha = \omega \cdot R$

    Выразим ускорение: $\vec{a} = \frac{d\vec{v}}{dt} = [\frac{d\vec{\omega}}{dt} \vec{r}] + [\vec{\omega} \frac{d\vec{r}}{dt}] = [\vec{\beta}\vec{r}] + [\vec{\omega}\vec{v}] = \vec{a}_\tau + \vec{a}_n$

    $\vec{a}_\tau$ называют тангенциальным ускорением (напраленным по касательной), $\vec{a}_n$ - нормальным (направленным к центру)

    $a_\tau = \beta \cdot r \cdot \sin\alpha = \beta \cdot R$

    Перемещение, путь, скорость:

    \begin{multicols}{2}

        $d\vec{r} = [d\vec{\varphi} \vec{\rho}] (\vec{\rho}\text{ - вектор радиуса окружности})$

        $dr = d\varphi \cdot R$

        $S = \varphi \cdot R$

        $\vec{v} = [\vec{\omega} \vec{\rho}]$

        $v = \omega \cdot R$

    \end{multicols}

    Ускорение: $\vec{a} = [\vec{\beta}\vec{r}] + [\vec{\omega}\vec{v}]$

    \begin{multicols}{3}

        $\vec{a}_\tau = [\vec{\beta}\vec{r}]$

        $a_\tau = \beta \cdot R$

        $\vec{a}_n = [\vec{\omega}\vec{v}] = [\vec{\omega}[\vec{\omega}\vec{\rho}]]$

        $a_n = \omega^2 R = \frac{1}{R} v^2$

        $T = \frac{2\pi}{\omega} = \frac{1}{\nu}$ - период

        $\nu = \frac{\omega}{2\pi} = \frac{1}{T}$ - частота

    \end{multicols}

    Плоское движение - движение твердого тела, при котором каждая его точка движется в плоскости,
    параллельной некоторой неподвижной в данной системе отсчета плоскости

    $\vec{r} = \vec{r}_0 + \vec{r}^\prime$

    $d\vec{r} = d\vec{r}_0 + d\vec{r}^\prime = d\vec{r}_0 + [d\vec{\varphi}\vec{r}]$

    $\vec{v} = \vec{v}_0 + [\vec{\omega}\vec{r}]$

    $\vec{v}_C$ - скорость центра колеса относительно точки отсчета

    $\vec{v}_{\text{вр}}$ - скорость точек колеса относительное его центра

    \Def Динамика - раздел механики, изучающий причины, вызывающие движение тел

    1687 г. - законы Ньютона, основа классической механики (механики Ньютона), обобщение большего количества опытов (Г. Галилей)

    Классическая механика - частный случай  1) СТО при скоростях много меньших скорости света $v \ll c$;
    2) квантовой механики при массах, много больших массы атома

    В динамике существуют различия между системами отсчета и преимущества одних СО над другими.

    Существуют такие системы отсчета, относительно которых свободное тело (тело, на которое не действуют другие тела) движется равномерно
    и прямолинейно или находится в состоянии покоя. Таким системы называются инерциальными (ИСО)

%    Земля $a_n = 3.4 \frac{\text{см}}{\text{с}^2}$
%
%    Центр Земли $a_n = 0.6 \frac{\text{см}}{\text{с}^2}$
%
%    Земля $a_n = 3 \cdot 10^{-8} \frac{\text{см}}{\text{с}^2}$
    % for later use i think

    \mediumvspace

    \textbf{Принцип относительности Галилея:}

    Любая СО, движущаяся с постоянной скоростью относительно ИСО, также является ИСО. Тогда справедливо любое из этих утверждений:

    \begin{enumerate}
        \item все ИСО эквивалентны друг другу по своим механическим свойствам
        \item во всех ИСО свойства пространства и времени одинаковы
        \item законы механики одинаковы во всех ИСО
    \end{enumerate}

    % сложное движение земли
    % перечитываю это и не понимаю, че было

    Преобразования Галилея - преобразования координат при переходе от одной ИСО к другой

    $K, K^\prime$ - ИСО

    $\vec{V}$ - скорость, с которой движется СО $K^\prime$ относительно $K$
    $t = t^\prime$

    $\vec{r} = \vec{r}^\prime + \vec{V}t$

    $\vec{c} = \vec{v}^\prime + \vec{V}$

    $\vec{a} = \vec{a}^\prime$

    \Def Сила - физическая величина, определяющая количественную характеристику и напраление воздействия, оказываемого на данное тело
    со стороны других тел.

    Силы условно можно разделить на силы, возникающие при непосредственном контакте (силы трения, давления) и на силы,
    возникающие через поля (электрические, гравитационные).

    \Def Инертная масса - мера инертности тела, то есть способности тела сохранять свою скорость при движении

    \Def Гравитационная масса - мера гравитацонного взаимодействия, величина, определяющая вес тел.

    $m_{\text{ин}} = m_{\text{гр}}$ с точностью до $10^{-13}$ кг

    В классической механике 1) масса - величина аддитивная ($m_1 + m_2 + \dots = m$); 2) $m = const$


    \subsection{Законы Ньютона}


    \begin{tcolorbox}[colframe=green!25, colback=green!10, title=\textbf{I закон Ньютона}, coltitle=black]
        Существуют такие системы отсчёта, называемые инерциальными, относительно которых материальные точки, когда на них не действуют никакие силы (или действуют силы взаимно уравновешенные), находятся в состоянии покоя или равномерного прямолинейного движения.
    \end{tcolorbox}


    \begin{tcolorbox}[colframe=green!25, colback=green!10, title=\textbf{II закон Ньютона}, coltitle=black]
        Ускорение тела пропорционально действующей на него силе и обратно пропорционально его массе $\vec{a} = \frac{\vec{F}}{m}$
    \end{tcolorbox}

    Под равнодействующей всех сил понимают векторную сумму всех сил, действующих на тело (принцип суперпозиции)

    $\vec{F} = \frac{d\vec{p}}{dt}$ - II закон в импульсной (дифференциальной) форме

    \begin{tcolorbox}[colframe=green!25, colback=green!10, title=\textbf{III закон Ньютона}, coltitle=black]
        Силы, с которыми два тела действуют друг на друга равны по модулю и направлены в противоположные стороны $\vec{F}_{12} = -\vec{F}_{21}$
    \end{tcolorbox}

    Закон Гука: $F = k|\Delta l|$ - сила упругости пропорциональна изменению длины тела

    Акселерометр - прибор, измеряющий ускорение, точнее проекцию кажущегося ускорения.

    Акселерометр использует II закон Ньютона ($mg - k\Delta l = ma$) во всех трех осях, что позволяет
    измерение ускорения в трех направлениях. Акселерометр используется в автомобилях, авиации, телефонах,
    игровых контроллерах, компьютерах (защита жесткого диска). Сейчас акселерометры изготавливаются
    в размерах от 20 мкм до 1 мм из кремния


\end{document}
