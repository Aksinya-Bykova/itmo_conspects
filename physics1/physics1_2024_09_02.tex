\documentclass[12pt]{article}
\usepackage{preamble}

\pagestyle{fancy}
\fancyhead[LO,LE]{Физические основы компьютерных \\ и сетевых технологий}
\fancyhead[CO,CE]{02.09.2024}
\fancyhead[RO,RE]{Лекции Музыченко Я. Б.}

\fancyfoot[L]{\scriptsize исходники найдутся тут: \\ \url{https://github.com/pelmesh619/itmo_conspects} \Cat}

\begin{document}
    \section{0. Вводная лекция}

    Задается вопрос: зачем обучающимся программистам нужна физика в учебном плане?

    Приводятся цитаты Л. Богуславского, одного из крупнейших IT инвесторов, и Б. Страуструпа, которые считают,
    что такие фундаментальные дисциплины, как математика, физика, иностранный язык, способствуют развитию
    мышления человека

    Такие компании, как Bell Labs и IBM создали прорывные изобретения в области физики, на основе которых
    построены компьютерные технологии

    В 3-ем семестре курс физики будет состоять из классической механики и основ электричества

    В 4-ом семестре будут темы магнетизма, колебаний, волн и волновых процессов

    В 5-ом семестре будут рассматриваться оптика, основы квантовой физики и квантовые вычисления

    Занятия состоят из лекций, практических и лабораторных занятий.
    Всего в 3-ем семестре будут 5 лабораторных работ


    \section{1. Современная физическая картина мира. Кинематика материальной точки}

    \textbf{Физика} - раздел естествознания, изучающий свойства и формы движения материи.
    Под материей понимают вещество и поля.

    Научный метод: сначала проводятся наблюдения и эксперименты, из которых выдвигается гипотеза и ищется
    адекватная математическая модель, эта гипотеза проверяется, и если она подтверждается,
    то формируется \textit{теория}

    Пример - открытие Нептуна: в 1781-1845 годах наблюдались аномалии в движении Урана, в 1845 проведение расчетов
    координат новой планеты, а в 1846 обнаружилась новая планета

    Принцип соответствия (Н. Бор, 1923 г.) - каждая новая теория должна включать предыдущую как частный случай

    Изучаемые объекты: вселенная, галактики, звездные системы и планеты, экосистемы, макротела, молекулы, атомы, ядра,
    элементарные частицы

    Всего в физике существуют 4 фундаментальных взаимодействия:

    \begin{tabular}{c|c|c}
        Взаимодействие   & Квант поля & Область взаимодействия                        \\ \hline
        Гравитационное   & гравитон   & масса                                         \\
        Электромагнитное & фотон      & все заряженные частицы, атомы, электротехника \\
        Слабое           & бозон      & радиоактивный распад                          \\
        Сильное          & глюон      & атомные ядра, фундаментальные частицы         \\
    \end{tabular}

    Механика - раздел физики, изучающий механическое движение, то есть движение тел в пространстве и времени.
    Механическое движение тел ОТНОСИТЕЛЬНО.

    \begin{tabular}{c|c|c|}
        & $\ll 3 \cdot 10^8$ м/с & $\approx 3 \cdot 10^8$ м/с \\ \hline
        $\gg 1$ нм & Классическая           & Релятивистская             \\ \hline
        $\ll 1$ нм & Квантовая              & Квантовая теория поля      \\ \hline
    \end{tabular}

    Материальная точка - тело, размерами которого можно пренебречь в условиях данной задачи

    Абсолютно твердое тело (АТТ) - система материальных точек, расстояние между которыми не меняется
    в процессе движения (деформации в процессе движения пренебрежимо малы)

    Тело отсчета - тело, относительно которого определяется положение других тел в пространстве

    Система отсчета - совокупность тела отсчета, связанной с ним системы координат и синхронизированных между собой часов

    % векторный, координатный и естественный способы

    Степени свободы - число независимых скалярных величин, однозначно определяющих положение тела в пространстве

    Материальная точка: $3$ степени свободы

    Система N материальных точек: $3N$ степени свободы

    АТТ: $6$ степеней свободы

    Система единиц (\deutscht{le System International d'unites}), 1960

    $[t] = $ с \quad $[S, l] = $ м

    7 основных единиц:

    $[S] = $ м \quad $[T] = $ К

    $[m] = $ кг \quad $[\nu] = $ моль

    $[t] = $ с \quad $[l] = $ Кд

    $[q] = $ Кл

    Изначально все физические единицы основывались на материальных предметов, из-за которых точности единиц была низкой,
    но недавно все единицы были переопределены на основе физических констант.

    В природе нет абсолютно точных вычислений. Измерение любой физической величины без погрешности не имеет смысла!

\end{document}
